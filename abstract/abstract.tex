\input{../settings/boilerplate}

\prelimheaders

\begin{abstract}
  The most pressing issues of our time are all characterized by sudden regime shifts: the collapse of marine fisheries or stock-markets, the overthrow of governments, shifts in global climate.  Regime shifts, or sudden transitions in dynamical behavior of a system, underly many important phenomena in ecological and evolutionary problems. How do they arise? How can we identify when a shift has occurred? Can we forecast these shifts? Here I address each of these central questions in the context of a particular system.  First, I show how stochasticity in eco-evolutionary dynamics can give rise two different domains, or regimes, governing the behavior of evolutionary trajectories~\citep{Boettiger2010}. In the next chapter, I turn to the question of identifying evolutionary shifts from data using phylogenetic trees and morphological trait data of extant species~\citep{Boettiger2012}.  In the last chapter, I adapt the approach of the previous section which allowed me to quantify the information available in a given data set that could detect a shift into an approach for detecting regime shifts in ecological time series data before the occur~\citep{Boettiger2012b}.  

 \end{abstract}

\input{../settings/boilerplate}
