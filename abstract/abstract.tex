% BOILERPLATE - Allows for individual chapters to be compiled.
%
% Usage:
%    % BOILERPLATE - Allows for individual chapters to be compiled.
%
% Usage:
%    % BOILERPLATE - Allows for individual chapters to be compiled.
%
% Usage:
%    \input{../settings/boilerplate} % place at start and end of chapter
% 

\def\precommands{%
    \input{../settings/phdsetup}
    \def\preinserted{}
    \begin{document}
}

\def\postcommands{%
    \singlespacing
    \phantomsection
%    \bibliographystyle{../bibliography/expanded}
   \bibliographystyle{elsarticle-harv}
   \bibliography{../bibliography/references}
    \enddocument
}

\def\autoinsert{%
    \ifx\preinserted\undefined
        \expandafter\precommands
    \else
        \expandafter\postcommands
    \fi
}

\ifx\master\undefined\expandafter\autoinsert\fi

 % place at start and end of chapter
% 

\def\precommands{%
    % [ USER VARIABLES ]

\def\PHDTITLE {Regime shifts in ecology and evolution}
\def\PHDAUTHOR{Carl Boettiger}
\def\PHDSCHOOL{University of California, Davis}

\def\PHDMONTH {September}
\def\PHDYEAR  {2012}
\def\PHDDEPT {Center for Population Biology}

\def\BSSCHOOL {Princeton}
\def\BSYEAR   {2007}

\def\PHDCOMMITTEEA{Alan Hastings}
\def\PHDCOMMITTEEB{Peter Wainwright}
\def\PHDCOMMITTEEC{Brian Moore}

% [ GLOBAL SETUP ]

\documentclass[letterpaper,oneside,11pt]{report}


% [ CARL BOETTIGER`S CUSTOM COMMANDS, LIBRARIES, ETC ]

\usepackage{subfigure}
\usepackage[sort&compress]{natbib}
\usepackage{color}
\usepackage{fancyvrb}
\usepackage{ctable}


\usepackage{silence}
\WarningFilter{amsmath}{Underfull}     

%\newcommand{\argmax}{\operatorname{argmax}}
\newcommand{\ud}{\mathrm{d}}


\usepackage{calc}
\usepackage{breakcites}
\usepackage[newcommands]{ragged2e}
\usepackage{appendix}
\usepackage{comment}
\usepackage{xifthen}

\usepackage{graphicx}
\usepackage{epstopdf}


\renewenvironment{abstract}{\chapter*{Abstract}}{}
\renewcommand{\bibname}{Bibliography}
\renewcommand{\contentsname}{Table of Contents}

\makeatletter
\renewcommand{\@biblabel}[1]{\textsc{#1}}
\makeatother

% [ FONT SETTINGS ]

\usepackage[T1]{fontenc}
\usepackage{libertine}

\usepackage[tbtags, intlimits, namelimits]{amsmath}
\usepackage{amsthm}
\usepackage{amssymb}
\usepackage{amsfonts}



% [ PAGE LAYOUT ]

\usepackage{geometry}
\geometry{lmargin = 1.5in}
\geometry{rmargin = 1.0in}
\geometry{tmargin = 1.0in}
\geometry{bmargin = 1.0in}

% [ PDF SETTINGS ]

\usepackage[final]{hyperref}
\hypersetup{
    breaklinks  = {true},
    colorlinks  = {true},
    linktocpage = {false},
    linkcolor   = {blue},
    citecolor   = {black},
    urlcolor    = {black},
    plainpages  = {false},
    pageanchor  = {true},
    pdfauthor   = {\PHDAUTHOR},
    pdftitle    = {\PHDTITLE},
    pdfsubject  = {Dissertation, \PHDSCHOOL},
    pdfcreator  = {},
    pdfkeywords = {},
    pdfproducer = {}
}
\urlstyle{same}

% [ LETTER SPACING ]

\usepackage[final]{microtype}
\microtypesetup{protrusion=compatibility}
\microtypesetup{expansion=false}

\newcommand{\upper}[1]{\MakeUppercase{#1}}
\let\lsscshape\scshape

\ifcase\pdfoutput\else\microtypesetup{letterspace=15}
\renewcommand{\scshape}{\lsscshape\lsstyle}
\renewcommand{\upper}[1]{\textls[50]{\MakeUppercase{#1}}}\fi

% [ LINE SPACING ]

\usepackage[doublespacing]{setspace}
\renewcommand{\displayskipstretch}{0.75}

\setlength{\parskip   }{0em}
\setlength{\parindent }{2em}

% [ TABLE FORMATTING ]

\usepackage{booktabs}
\usepackage{multirow}
\usepackage{dcolumn}
\setlength{\heavyrulewidth}{1.5\arrayrulewidth}
\setlength{\lightrulewidth}{1.0\arrayrulewidth}
\setlength{\doublerulesep }{2.0\arrayrulewidth}

% [ SECTION FORMATTING ]

\usepackage[largestsep,nobottomtitles*]{titlesec}
\renewcommand{\bottomtitlespace}{0.75in}

\titleformat{\chapter}[display]%
    {\bfseries\huge\singlespacing}%
    {\filleft\textsc{\LARGE \chaptertitlename\ \thechapter}}%
    {-0.2ex}{\titlerule[3pt]\vspace{0.2ex}}[]

\titleformat{\section}{\LARGE}{\thesection\hspace{0.5em}}{0ex}{}
\titleformat{\subsection}{\Large}{\thesubsection\hspace{0.5em}}{0ex}{}
\titleformat{\subsubsection}{\large}{\thesubsubsection\hspace{0.5em}}{0ex}{}

\titlespacing*{\chapter}{0em}{6ex}{4ex plus 2ex minus 0ex}
\titlespacing*{\section}{0em}{2ex plus 3ex minus 1ex}{0.5ex plus 0.5ex minus 0.5ex}
\titlespacing*{\subsection}{0ex}{2ex plus 3ex minus 1ex}{0ex}
\titlespacing*{\subsubsection}{0ex}{2ex plus 0ex minus 1ex}{0ex}

% [ HEADER SETTINGS ]

\usepackage{fancyhdr}

\setlength{\headheight}{\normalbaselineskip}
\setlength{\footskip  }{0.5in}
\setlength{\headsep   }{0.5in-\headheight}

\fancyheadoffset[R]{0.5in}
\renewcommand{\headrulewidth}{0pt}
\renewcommand{\footrulewidth}{0pt}

\newcommand{\pagebox}{\parbox[r][\headheight][t]{0.5in}{\hspace\fill\thepage}}

\newcommand{\prelimheaders}{\ifx\prelim\undefined\renewcommand{\thepage}{\textit{\roman{page}}}\fancypagestyle{plain}{\fancyhf{}\fancyfoot[L]{\makebox[\textwidth-0.5in]{\thepage}}}\pagestyle{plain}\def\prelim{}\fi}

\newcommand{\normalheaders}{\renewcommand{\thepage}{\arabic{page}}\fancypagestyle{plain}{\fancyhf{}\fancyhead[R]{\pagebox}}\pagestyle{plain}}

\normalheaders{}

% [ CUSTOM COMMANDS ]

\newcommand{\signaturebox}[1]{\multicolumn{1}{p{4in}}{\vspace{3ex}}\\\midrule #1\\}

% Redefine AMS proof environment to have itshape
% Note: This environment automatically adds \qed at the end. If your proof
% ends in a math environment, the \qed is placed, undesirably, on a new line.
% To prevent that, insert \qedhere inside the math environment.
\makeatletter
\renewenvironment{proof}[1][\proofname]{%
\par\pushQED{\qed}\normalfont%
\topsep6\p@\@plus6\p@\relax\trivlist%
\item[\hskip\labelsep\bfseries#1\@addpunct{.}]\itshape\ignorespaces}{%
\popQED\endtrivlist\@endpefalse}%
\makeatother

% TUGboat, Volume 0 (2001), No. 0
% http://math.arizona.edu/~aprl/publications/mathclap/perlis_mathclap_24Jun2003.pdf
% For comparison, here are the existing overlap macros:
% \def\llap#1{\hbox to 0pt{\hss#1}}
% \def\rlap#1{\hbox to 0pt{#1\hss}}
\def\clap#1{\hbox to 0pt{\hss#1\hss}}
\def\mathllap{\mathpalette\mathllapinternal}
\def\mathrlap{\mathpalette\mathrlapinternal}
\def\mathclap{\mathpalette\mathclapinternal}
\def\mathllapinternal#1#2{%
\llap{$\mathsurround=0pt#1{#2}$}}
\def\mathrlapinternal#1#2{%
\rlap{$\mathsurround=0pt#1{#2}$}}
\def\mathclapinternal#1#2{%
\clap{$\mathsurround=0pt#1{#2}$}}

\newcommand{\alert}[1]{\textbf{\textcolor{red}{#1}}}




% [ Code blocks ]
%\DefineShortVerb[commandchars=\\\{\}]{\|}
\DefineVerbatimEnvironment{Highlighting}{Verbatim}{commandchars=\\\{\}}
% Add ',fontsize=\small' for more characters per line
\newenvironment{Shaded}{}{}
\newcommand{\KeywordTok}[1]{\textcolor[rgb]{0.00,0.44,0.13}{\textbf{{#1}}}}
\newcommand{\DataTypeTok}[1]{\textcolor[rgb]{0.56,0.13,0.00}{{#1}}}
\newcommand{\DecValTok}[1]{\textcolor[rgb]{0.25,0.63,0.44}{{#1}}}
\newcommand{\BaseNTok}[1]{\textcolor[rgb]{0.25,0.63,0.44}{{#1}}}
\newcommand{\FloatTok}[1]{\textcolor[rgb]{0.25,0.63,0.44}{{#1}}}
\newcommand{\CharTok}[1]{\textcolor[rgb]{0.25,0.44,0.63}{{#1}}}
\newcommand{\StringTok}[1]{\textcolor[rgb]{0.25,0.44,0.63}{{#1}}}
\newcommand{\CommentTok}[1]{\textcolor[rgb]{0.38,0.63,0.69}{\textit{{#1}}}}
\newcommand{\OtherTok}[1]{\textcolor[rgb]{0.00,0.44,0.13}{{#1}}}
\newcommand{\AlertTok}[1]{\textcolor[rgb]{1.00,0.00,0.00}{\textbf{{#1}}}}
\newcommand{\FunctionTok}[1]{\textcolor[rgb]{0.02,0.16,0.49}{{#1}}}
\newcommand{\RegionMarkerTok}[1]{{#1}}
\newcommand{\ErrorTok}[1]{\textcolor[rgb]{1.00,0.00,0.00}{\textbf{{#1}}}}
\newcommand{\NormalTok}[1]{{#1}}



    \def\preinserted{}
    \begin{document}
}

\def\postcommands{%
    \singlespacing
    \phantomsection
%    \bibliographystyle{../bibliography/expanded}
   \bibliographystyle{elsarticle-harv}
   \bibliography{../bibliography/references}
    \enddocument
}

\def\autoinsert{%
    \ifx\preinserted\undefined
        \expandafter\precommands
    \else
        \expandafter\postcommands
    \fi
}

\ifx\master\undefined\expandafter\autoinsert\fi

 % place at start and end of chapter
% 

\def\precommands{%
    % [ USER VARIABLES ]

\def\PHDTITLE {Regime shifts in ecology and evolution}
\def\PHDAUTHOR{Carl Boettiger}
\def\PHDSCHOOL{University of California, Davis}

\def\PHDMONTH {September}
\def\PHDYEAR  {2012}
\def\PHDDEPT {Center for Population Biology}

\def\BSSCHOOL {Princeton}
\def\BSYEAR   {2007}

\def\PHDCOMMITTEEA{Alan Hastings}
\def\PHDCOMMITTEEB{Peter Wainwright}
\def\PHDCOMMITTEEC{Brian Moore}

% [ GLOBAL SETUP ]

\documentclass[letterpaper,oneside,11pt]{report}


% [ CARL BOETTIGER`S CUSTOM COMMANDS, LIBRARIES, ETC ]

\usepackage{subfigure}
\usepackage[sort&compress]{natbib}
\usepackage{color}
\usepackage{fancyvrb}
\usepackage{ctable}


\usepackage{silence}
\WarningFilter{amsmath}{Underfull}     

%\newcommand{\argmax}{\operatorname{argmax}}
\newcommand{\ud}{\mathrm{d}}


\usepackage{calc}
\usepackage{breakcites}
\usepackage[newcommands]{ragged2e}
\usepackage{appendix}
\usepackage{comment}
\usepackage{xifthen}

\usepackage{graphicx}
\usepackage{epstopdf}


\renewenvironment{abstract}{\chapter*{Abstract}}{}
\renewcommand{\bibname}{Bibliography}
\renewcommand{\contentsname}{Table of Contents}

\makeatletter
\renewcommand{\@biblabel}[1]{\textsc{#1}}
\makeatother

% [ FONT SETTINGS ]

\usepackage[T1]{fontenc}
\usepackage{libertine}

\usepackage[tbtags, intlimits, namelimits]{amsmath}
\usepackage{amsthm}
\usepackage{amssymb}
\usepackage{amsfonts}



% [ PAGE LAYOUT ]

\usepackage{geometry}
\geometry{lmargin = 1.5in}
\geometry{rmargin = 1.0in}
\geometry{tmargin = 1.0in}
\geometry{bmargin = 1.0in}

% [ PDF SETTINGS ]

\usepackage[final]{hyperref}
\hypersetup{
    breaklinks  = {true},
    colorlinks  = {true},
    linktocpage = {false},
    linkcolor   = {blue},
    citecolor   = {black},
    urlcolor    = {black},
    plainpages  = {false},
    pageanchor  = {true},
    pdfauthor   = {\PHDAUTHOR},
    pdftitle    = {\PHDTITLE},
    pdfsubject  = {Dissertation, \PHDSCHOOL},
    pdfcreator  = {},
    pdfkeywords = {},
    pdfproducer = {}
}
\urlstyle{same}

% [ LETTER SPACING ]

\usepackage[final]{microtype}
\microtypesetup{protrusion=compatibility}
\microtypesetup{expansion=false}

\newcommand{\upper}[1]{\MakeUppercase{#1}}
\let\lsscshape\scshape

\ifcase\pdfoutput\else\microtypesetup{letterspace=15}
\renewcommand{\scshape}{\lsscshape\lsstyle}
\renewcommand{\upper}[1]{\textls[50]{\MakeUppercase{#1}}}\fi

% [ LINE SPACING ]

\usepackage[doublespacing]{setspace}
\renewcommand{\displayskipstretch}{0.75}

\setlength{\parskip   }{0em}
\setlength{\parindent }{2em}

% [ TABLE FORMATTING ]

\usepackage{booktabs}
\usepackage{multirow}
\usepackage{dcolumn}
\setlength{\heavyrulewidth}{1.5\arrayrulewidth}
\setlength{\lightrulewidth}{1.0\arrayrulewidth}
\setlength{\doublerulesep }{2.0\arrayrulewidth}

% [ SECTION FORMATTING ]

\usepackage[largestsep,nobottomtitles*]{titlesec}
\renewcommand{\bottomtitlespace}{0.75in}

\titleformat{\chapter}[display]%
    {\bfseries\huge\singlespacing}%
    {\filleft\textsc{\LARGE \chaptertitlename\ \thechapter}}%
    {-0.2ex}{\titlerule[3pt]\vspace{0.2ex}}[]

\titleformat{\section}{\LARGE}{\thesection\hspace{0.5em}}{0ex}{}
\titleformat{\subsection}{\Large}{\thesubsection\hspace{0.5em}}{0ex}{}
\titleformat{\subsubsection}{\large}{\thesubsubsection\hspace{0.5em}}{0ex}{}

\titlespacing*{\chapter}{0em}{6ex}{4ex plus 2ex minus 0ex}
\titlespacing*{\section}{0em}{2ex plus 3ex minus 1ex}{0.5ex plus 0.5ex minus 0.5ex}
\titlespacing*{\subsection}{0ex}{2ex plus 3ex minus 1ex}{0ex}
\titlespacing*{\subsubsection}{0ex}{2ex plus 0ex minus 1ex}{0ex}

% [ HEADER SETTINGS ]

\usepackage{fancyhdr}

\setlength{\headheight}{\normalbaselineskip}
\setlength{\footskip  }{0.5in}
\setlength{\headsep   }{0.5in-\headheight}

\fancyheadoffset[R]{0.5in}
\renewcommand{\headrulewidth}{0pt}
\renewcommand{\footrulewidth}{0pt}

\newcommand{\pagebox}{\parbox[r][\headheight][t]{0.5in}{\hspace\fill\thepage}}

\newcommand{\prelimheaders}{\ifx\prelim\undefined\renewcommand{\thepage}{\textit{\roman{page}}}\fancypagestyle{plain}{\fancyhf{}\fancyfoot[L]{\makebox[\textwidth-0.5in]{\thepage}}}\pagestyle{plain}\def\prelim{}\fi}

\newcommand{\normalheaders}{\renewcommand{\thepage}{\arabic{page}}\fancypagestyle{plain}{\fancyhf{}\fancyhead[R]{\pagebox}}\pagestyle{plain}}

\normalheaders{}

% [ CUSTOM COMMANDS ]

\newcommand{\signaturebox}[1]{\multicolumn{1}{p{4in}}{\vspace{3ex}}\\\midrule #1\\}

% Redefine AMS proof environment to have itshape
% Note: This environment automatically adds \qed at the end. If your proof
% ends in a math environment, the \qed is placed, undesirably, on a new line.
% To prevent that, insert \qedhere inside the math environment.
\makeatletter
\renewenvironment{proof}[1][\proofname]{%
\par\pushQED{\qed}\normalfont%
\topsep6\p@\@plus6\p@\relax\trivlist%
\item[\hskip\labelsep\bfseries#1\@addpunct{.}]\itshape\ignorespaces}{%
\popQED\endtrivlist\@endpefalse}%
\makeatother

% TUGboat, Volume 0 (2001), No. 0
% http://math.arizona.edu/~aprl/publications/mathclap/perlis_mathclap_24Jun2003.pdf
% For comparison, here are the existing overlap macros:
% \def\llap#1{\hbox to 0pt{\hss#1}}
% \def\rlap#1{\hbox to 0pt{#1\hss}}
\def\clap#1{\hbox to 0pt{\hss#1\hss}}
\def\mathllap{\mathpalette\mathllapinternal}
\def\mathrlap{\mathpalette\mathrlapinternal}
\def\mathclap{\mathpalette\mathclapinternal}
\def\mathllapinternal#1#2{%
\llap{$\mathsurround=0pt#1{#2}$}}
\def\mathrlapinternal#1#2{%
\rlap{$\mathsurround=0pt#1{#2}$}}
\def\mathclapinternal#1#2{%
\clap{$\mathsurround=0pt#1{#2}$}}

\newcommand{\alert}[1]{\textbf{\textcolor{red}{#1}}}




% [ Code blocks ]
%\DefineShortVerb[commandchars=\\\{\}]{\|}
\DefineVerbatimEnvironment{Highlighting}{Verbatim}{commandchars=\\\{\}}
% Add ',fontsize=\small' for more characters per line
\newenvironment{Shaded}{}{}
\newcommand{\KeywordTok}[1]{\textcolor[rgb]{0.00,0.44,0.13}{\textbf{{#1}}}}
\newcommand{\DataTypeTok}[1]{\textcolor[rgb]{0.56,0.13,0.00}{{#1}}}
\newcommand{\DecValTok}[1]{\textcolor[rgb]{0.25,0.63,0.44}{{#1}}}
\newcommand{\BaseNTok}[1]{\textcolor[rgb]{0.25,0.63,0.44}{{#1}}}
\newcommand{\FloatTok}[1]{\textcolor[rgb]{0.25,0.63,0.44}{{#1}}}
\newcommand{\CharTok}[1]{\textcolor[rgb]{0.25,0.44,0.63}{{#1}}}
\newcommand{\StringTok}[1]{\textcolor[rgb]{0.25,0.44,0.63}{{#1}}}
\newcommand{\CommentTok}[1]{\textcolor[rgb]{0.38,0.63,0.69}{\textit{{#1}}}}
\newcommand{\OtherTok}[1]{\textcolor[rgb]{0.00,0.44,0.13}{{#1}}}
\newcommand{\AlertTok}[1]{\textcolor[rgb]{1.00,0.00,0.00}{\textbf{{#1}}}}
\newcommand{\FunctionTok}[1]{\textcolor[rgb]{0.02,0.16,0.49}{{#1}}}
\newcommand{\RegionMarkerTok}[1]{{#1}}
\newcommand{\ErrorTok}[1]{\textcolor[rgb]{1.00,0.00,0.00}{\textbf{{#1}}}}
\newcommand{\NormalTok}[1]{{#1}}



    \def\preinserted{}
    \begin{document}
}

\def\postcommands{%
    \singlespacing
    \phantomsection
%    \bibliographystyle{../bibliography/expanded}
   \bibliographystyle{elsarticle-harv}
   \bibliography{../bibliography/references}
    \enddocument
}

\def\autoinsert{%
    \ifx\preinserted\undefined
        \expandafter\precommands
    \else
        \expandafter\postcommands
    \fi
}

\ifx\master\undefined\expandafter\autoinsert\fi



\prelimheaders

\begin{abstract}
  The most pressing issues of our time are all characterized by sudden regime shifts: the collapse of marine fisheries or stock-markets, the overthrow of governments, shifts in global climate.  Regime shifts, or sudden transitions in dynamical behavior of a system, underly many important phenomena in ecological and evolutionary problems. How do they arise? How can we identify when a shift has occurred? Can we forecast these shifts? Here I address each of these central questions in the context of a particular system.  First, I show how stochasticity in eco-evolutionary dynamics can give rise two different domains, or regimes, governing the behavior of evolutionary trajectories~\citep{Boettiger2010}. In the next chapter, I turn to the question of identifying evolutionary shifts from data using phylogenetic trees and morphological trait data of extant species~\citep{Boettiger2012}.  In the last chapter, I adapt the approach of the previous section which allowed me to quantify the information available in a given data set that could detect a shift into an approach for detecting regime shifts in ecological time series data before the occur~\citep{Boettiger2012b}.  

 \end{abstract}

% BOILERPLATE - Allows for individual chapters to be compiled.
%
% Usage:
%    % BOILERPLATE - Allows for individual chapters to be compiled.
%
% Usage:
%    % BOILERPLATE - Allows for individual chapters to be compiled.
%
% Usage:
%    \input{../settings/boilerplate} % place at start and end of chapter
% 

\def\precommands{%
    \input{../settings/phdsetup}
    \def\preinserted{}
    \begin{document}
}

\def\postcommands{%
    \singlespacing
    \phantomsection
%    \bibliographystyle{../bibliography/expanded}
   \bibliographystyle{elsarticle-harv}
   \bibliography{../bibliography/references}
    \enddocument
}

\def\autoinsert{%
    \ifx\preinserted\undefined
        \expandafter\precommands
    \else
        \expandafter\postcommands
    \fi
}

\ifx\master\undefined\expandafter\autoinsert\fi

 % place at start and end of chapter
% 

\def\precommands{%
    % [ USER VARIABLES ]

\def\PHDTITLE {Regime shifts in ecology and evolution}
\def\PHDAUTHOR{Carl Boettiger}
\def\PHDSCHOOL{University of California, Davis}

\def\PHDMONTH {September}
\def\PHDYEAR  {2012}
\def\PHDDEPT {Center for Population Biology}

\def\BSSCHOOL {Princeton}
\def\BSYEAR   {2007}

\def\PHDCOMMITTEEA{Alan Hastings}
\def\PHDCOMMITTEEB{Peter Wainwright}
\def\PHDCOMMITTEEC{Brian Moore}

% [ GLOBAL SETUP ]

\documentclass[letterpaper,oneside,11pt]{report}


% [ CARL BOETTIGER`S CUSTOM COMMANDS, LIBRARIES, ETC ]

\usepackage{subfigure}
\usepackage[sort&compress]{natbib}
\usepackage{color}
\usepackage{fancyvrb}
\usepackage{ctable}


\usepackage{silence}
\WarningFilter{amsmath}{Underfull}     

%\newcommand{\argmax}{\operatorname{argmax}}
\newcommand{\ud}{\mathrm{d}}


\usepackage{calc}
\usepackage{breakcites}
\usepackage[newcommands]{ragged2e}
\usepackage{appendix}
\usepackage{comment}
\usepackage{xifthen}

\usepackage{graphicx}
\usepackage{epstopdf}


\renewenvironment{abstract}{\chapter*{Abstract}}{}
\renewcommand{\bibname}{Bibliography}
\renewcommand{\contentsname}{Table of Contents}

\makeatletter
\renewcommand{\@biblabel}[1]{\textsc{#1}}
\makeatother

% [ FONT SETTINGS ]

\usepackage[T1]{fontenc}
\usepackage{libertine}

\usepackage[tbtags, intlimits, namelimits]{amsmath}
\usepackage{amsthm}
\usepackage{amssymb}
\usepackage{amsfonts}



% [ PAGE LAYOUT ]

\usepackage{geometry}
\geometry{lmargin = 1.5in}
\geometry{rmargin = 1.0in}
\geometry{tmargin = 1.0in}
\geometry{bmargin = 1.0in}

% [ PDF SETTINGS ]

\usepackage[final]{hyperref}
\hypersetup{
    breaklinks  = {true},
    colorlinks  = {true},
    linktocpage = {false},
    linkcolor   = {blue},
    citecolor   = {black},
    urlcolor    = {black},
    plainpages  = {false},
    pageanchor  = {true},
    pdfauthor   = {\PHDAUTHOR},
    pdftitle    = {\PHDTITLE},
    pdfsubject  = {Dissertation, \PHDSCHOOL},
    pdfcreator  = {},
    pdfkeywords = {},
    pdfproducer = {}
}
\urlstyle{same}

% [ LETTER SPACING ]

\usepackage[final]{microtype}
\microtypesetup{protrusion=compatibility}
\microtypesetup{expansion=false}

\newcommand{\upper}[1]{\MakeUppercase{#1}}
\let\lsscshape\scshape

\ifcase\pdfoutput\else\microtypesetup{letterspace=15}
\renewcommand{\scshape}{\lsscshape\lsstyle}
\renewcommand{\upper}[1]{\textls[50]{\MakeUppercase{#1}}}\fi

% [ LINE SPACING ]

\usepackage[doublespacing]{setspace}
\renewcommand{\displayskipstretch}{0.75}

\setlength{\parskip   }{0em}
\setlength{\parindent }{2em}

% [ TABLE FORMATTING ]

\usepackage{booktabs}
\usepackage{multirow}
\usepackage{dcolumn}
\setlength{\heavyrulewidth}{1.5\arrayrulewidth}
\setlength{\lightrulewidth}{1.0\arrayrulewidth}
\setlength{\doublerulesep }{2.0\arrayrulewidth}

% [ SECTION FORMATTING ]

\usepackage[largestsep,nobottomtitles*]{titlesec}
\renewcommand{\bottomtitlespace}{0.75in}

\titleformat{\chapter}[display]%
    {\bfseries\huge\singlespacing}%
    {\filleft\textsc{\LARGE \chaptertitlename\ \thechapter}}%
    {-0.2ex}{\titlerule[3pt]\vspace{0.2ex}}[]

\titleformat{\section}{\LARGE}{\thesection\hspace{0.5em}}{0ex}{}
\titleformat{\subsection}{\Large}{\thesubsection\hspace{0.5em}}{0ex}{}
\titleformat{\subsubsection}{\large}{\thesubsubsection\hspace{0.5em}}{0ex}{}

\titlespacing*{\chapter}{0em}{6ex}{4ex plus 2ex minus 0ex}
\titlespacing*{\section}{0em}{2ex plus 3ex minus 1ex}{0.5ex plus 0.5ex minus 0.5ex}
\titlespacing*{\subsection}{0ex}{2ex plus 3ex minus 1ex}{0ex}
\titlespacing*{\subsubsection}{0ex}{2ex plus 0ex minus 1ex}{0ex}

% [ HEADER SETTINGS ]

\usepackage{fancyhdr}

\setlength{\headheight}{\normalbaselineskip}
\setlength{\footskip  }{0.5in}
\setlength{\headsep   }{0.5in-\headheight}

\fancyheadoffset[R]{0.5in}
\renewcommand{\headrulewidth}{0pt}
\renewcommand{\footrulewidth}{0pt}

\newcommand{\pagebox}{\parbox[r][\headheight][t]{0.5in}{\hspace\fill\thepage}}

\newcommand{\prelimheaders}{\ifx\prelim\undefined\renewcommand{\thepage}{\textit{\roman{page}}}\fancypagestyle{plain}{\fancyhf{}\fancyfoot[L]{\makebox[\textwidth-0.5in]{\thepage}}}\pagestyle{plain}\def\prelim{}\fi}

\newcommand{\normalheaders}{\renewcommand{\thepage}{\arabic{page}}\fancypagestyle{plain}{\fancyhf{}\fancyhead[R]{\pagebox}}\pagestyle{plain}}

\normalheaders{}

% [ CUSTOM COMMANDS ]

\newcommand{\signaturebox}[1]{\multicolumn{1}{p{4in}}{\vspace{3ex}}\\\midrule #1\\}

% Redefine AMS proof environment to have itshape
% Note: This environment automatically adds \qed at the end. If your proof
% ends in a math environment, the \qed is placed, undesirably, on a new line.
% To prevent that, insert \qedhere inside the math environment.
\makeatletter
\renewenvironment{proof}[1][\proofname]{%
\par\pushQED{\qed}\normalfont%
\topsep6\p@\@plus6\p@\relax\trivlist%
\item[\hskip\labelsep\bfseries#1\@addpunct{.}]\itshape\ignorespaces}{%
\popQED\endtrivlist\@endpefalse}%
\makeatother

% TUGboat, Volume 0 (2001), No. 0
% http://math.arizona.edu/~aprl/publications/mathclap/perlis_mathclap_24Jun2003.pdf
% For comparison, here are the existing overlap macros:
% \def\llap#1{\hbox to 0pt{\hss#1}}
% \def\rlap#1{\hbox to 0pt{#1\hss}}
\def\clap#1{\hbox to 0pt{\hss#1\hss}}
\def\mathllap{\mathpalette\mathllapinternal}
\def\mathrlap{\mathpalette\mathrlapinternal}
\def\mathclap{\mathpalette\mathclapinternal}
\def\mathllapinternal#1#2{%
\llap{$\mathsurround=0pt#1{#2}$}}
\def\mathrlapinternal#1#2{%
\rlap{$\mathsurround=0pt#1{#2}$}}
\def\mathclapinternal#1#2{%
\clap{$\mathsurround=0pt#1{#2}$}}

\newcommand{\alert}[1]{\textbf{\textcolor{red}{#1}}}




% [ Code blocks ]
%\DefineShortVerb[commandchars=\\\{\}]{\|}
\DefineVerbatimEnvironment{Highlighting}{Verbatim}{commandchars=\\\{\}}
% Add ',fontsize=\small' for more characters per line
\newenvironment{Shaded}{}{}
\newcommand{\KeywordTok}[1]{\textcolor[rgb]{0.00,0.44,0.13}{\textbf{{#1}}}}
\newcommand{\DataTypeTok}[1]{\textcolor[rgb]{0.56,0.13,0.00}{{#1}}}
\newcommand{\DecValTok}[1]{\textcolor[rgb]{0.25,0.63,0.44}{{#1}}}
\newcommand{\BaseNTok}[1]{\textcolor[rgb]{0.25,0.63,0.44}{{#1}}}
\newcommand{\FloatTok}[1]{\textcolor[rgb]{0.25,0.63,0.44}{{#1}}}
\newcommand{\CharTok}[1]{\textcolor[rgb]{0.25,0.44,0.63}{{#1}}}
\newcommand{\StringTok}[1]{\textcolor[rgb]{0.25,0.44,0.63}{{#1}}}
\newcommand{\CommentTok}[1]{\textcolor[rgb]{0.38,0.63,0.69}{\textit{{#1}}}}
\newcommand{\OtherTok}[1]{\textcolor[rgb]{0.00,0.44,0.13}{{#1}}}
\newcommand{\AlertTok}[1]{\textcolor[rgb]{1.00,0.00,0.00}{\textbf{{#1}}}}
\newcommand{\FunctionTok}[1]{\textcolor[rgb]{0.02,0.16,0.49}{{#1}}}
\newcommand{\RegionMarkerTok}[1]{{#1}}
\newcommand{\ErrorTok}[1]{\textcolor[rgb]{1.00,0.00,0.00}{\textbf{{#1}}}}
\newcommand{\NormalTok}[1]{{#1}}



    \def\preinserted{}
    \begin{document}
}

\def\postcommands{%
    \singlespacing
    \phantomsection
%    \bibliographystyle{../bibliography/expanded}
   \bibliographystyle{elsarticle-harv}
   \bibliography{../bibliography/references}
    \enddocument
}

\def\autoinsert{%
    \ifx\preinserted\undefined
        \expandafter\precommands
    \else
        \expandafter\postcommands
    \fi
}

\ifx\master\undefined\expandafter\autoinsert\fi

 % place at start and end of chapter
% 

\def\precommands{%
    % [ USER VARIABLES ]

\def\PHDTITLE {Regime shifts in ecology and evolution}
\def\PHDAUTHOR{Carl Boettiger}
\def\PHDSCHOOL{University of California, Davis}

\def\PHDMONTH {September}
\def\PHDYEAR  {2012}
\def\PHDDEPT {Center for Population Biology}

\def\BSSCHOOL {Princeton}
\def\BSYEAR   {2007}

\def\PHDCOMMITTEEA{Alan Hastings}
\def\PHDCOMMITTEEB{Peter Wainwright}
\def\PHDCOMMITTEEC{Brian Moore}

% [ GLOBAL SETUP ]

\documentclass[letterpaper,oneside,11pt]{report}


% [ CARL BOETTIGER`S CUSTOM COMMANDS, LIBRARIES, ETC ]

\usepackage{subfigure}
\usepackage[sort&compress]{natbib}
\usepackage{color}
\usepackage{fancyvrb}
\usepackage{ctable}


\usepackage{silence}
\WarningFilter{amsmath}{Underfull}     

%\newcommand{\argmax}{\operatorname{argmax}}
\newcommand{\ud}{\mathrm{d}}


\usepackage{calc}
\usepackage{breakcites}
\usepackage[newcommands]{ragged2e}
\usepackage{appendix}
\usepackage{comment}
\usepackage{xifthen}

\usepackage{graphicx}
\usepackage{epstopdf}


\renewenvironment{abstract}{\chapter*{Abstract}}{}
\renewcommand{\bibname}{Bibliography}
\renewcommand{\contentsname}{Table of Contents}

\makeatletter
\renewcommand{\@biblabel}[1]{\textsc{#1}}
\makeatother

% [ FONT SETTINGS ]

\usepackage[T1]{fontenc}
\usepackage{libertine}

\usepackage[tbtags, intlimits, namelimits]{amsmath}
\usepackage{amsthm}
\usepackage{amssymb}
\usepackage{amsfonts}



% [ PAGE LAYOUT ]

\usepackage{geometry}
\geometry{lmargin = 1.5in}
\geometry{rmargin = 1.0in}
\geometry{tmargin = 1.0in}
\geometry{bmargin = 1.0in}

% [ PDF SETTINGS ]

\usepackage[final]{hyperref}
\hypersetup{
    breaklinks  = {true},
    colorlinks  = {true},
    linktocpage = {false},
    linkcolor   = {blue},
    citecolor   = {black},
    urlcolor    = {black},
    plainpages  = {false},
    pageanchor  = {true},
    pdfauthor   = {\PHDAUTHOR},
    pdftitle    = {\PHDTITLE},
    pdfsubject  = {Dissertation, \PHDSCHOOL},
    pdfcreator  = {},
    pdfkeywords = {},
    pdfproducer = {}
}
\urlstyle{same}

% [ LETTER SPACING ]

\usepackage[final]{microtype}
\microtypesetup{protrusion=compatibility}
\microtypesetup{expansion=false}

\newcommand{\upper}[1]{\MakeUppercase{#1}}
\let\lsscshape\scshape

\ifcase\pdfoutput\else\microtypesetup{letterspace=15}
\renewcommand{\scshape}{\lsscshape\lsstyle}
\renewcommand{\upper}[1]{\textls[50]{\MakeUppercase{#1}}}\fi

% [ LINE SPACING ]

\usepackage[doublespacing]{setspace}
\renewcommand{\displayskipstretch}{0.75}

\setlength{\parskip   }{0em}
\setlength{\parindent }{2em}

% [ TABLE FORMATTING ]

\usepackage{booktabs}
\usepackage{multirow}
\usepackage{dcolumn}
\setlength{\heavyrulewidth}{1.5\arrayrulewidth}
\setlength{\lightrulewidth}{1.0\arrayrulewidth}
\setlength{\doublerulesep }{2.0\arrayrulewidth}

% [ SECTION FORMATTING ]

\usepackage[largestsep,nobottomtitles*]{titlesec}
\renewcommand{\bottomtitlespace}{0.75in}

\titleformat{\chapter}[display]%
    {\bfseries\huge\singlespacing}%
    {\filleft\textsc{\LARGE \chaptertitlename\ \thechapter}}%
    {-0.2ex}{\titlerule[3pt]\vspace{0.2ex}}[]

\titleformat{\section}{\LARGE}{\thesection\hspace{0.5em}}{0ex}{}
\titleformat{\subsection}{\Large}{\thesubsection\hspace{0.5em}}{0ex}{}
\titleformat{\subsubsection}{\large}{\thesubsubsection\hspace{0.5em}}{0ex}{}

\titlespacing*{\chapter}{0em}{6ex}{4ex plus 2ex minus 0ex}
\titlespacing*{\section}{0em}{2ex plus 3ex minus 1ex}{0.5ex plus 0.5ex minus 0.5ex}
\titlespacing*{\subsection}{0ex}{2ex plus 3ex minus 1ex}{0ex}
\titlespacing*{\subsubsection}{0ex}{2ex plus 0ex minus 1ex}{0ex}

% [ HEADER SETTINGS ]

\usepackage{fancyhdr}

\setlength{\headheight}{\normalbaselineskip}
\setlength{\footskip  }{0.5in}
\setlength{\headsep   }{0.5in-\headheight}

\fancyheadoffset[R]{0.5in}
\renewcommand{\headrulewidth}{0pt}
\renewcommand{\footrulewidth}{0pt}

\newcommand{\pagebox}{\parbox[r][\headheight][t]{0.5in}{\hspace\fill\thepage}}

\newcommand{\prelimheaders}{\ifx\prelim\undefined\renewcommand{\thepage}{\textit{\roman{page}}}\fancypagestyle{plain}{\fancyhf{}\fancyfoot[L]{\makebox[\textwidth-0.5in]{\thepage}}}\pagestyle{plain}\def\prelim{}\fi}

\newcommand{\normalheaders}{\renewcommand{\thepage}{\arabic{page}}\fancypagestyle{plain}{\fancyhf{}\fancyhead[R]{\pagebox}}\pagestyle{plain}}

\normalheaders{}

% [ CUSTOM COMMANDS ]

\newcommand{\signaturebox}[1]{\multicolumn{1}{p{4in}}{\vspace{3ex}}\\\midrule #1\\}

% Redefine AMS proof environment to have itshape
% Note: This environment automatically adds \qed at the end. If your proof
% ends in a math environment, the \qed is placed, undesirably, on a new line.
% To prevent that, insert \qedhere inside the math environment.
\makeatletter
\renewenvironment{proof}[1][\proofname]{%
\par\pushQED{\qed}\normalfont%
\topsep6\p@\@plus6\p@\relax\trivlist%
\item[\hskip\labelsep\bfseries#1\@addpunct{.}]\itshape\ignorespaces}{%
\popQED\endtrivlist\@endpefalse}%
\makeatother

% TUGboat, Volume 0 (2001), No. 0
% http://math.arizona.edu/~aprl/publications/mathclap/perlis_mathclap_24Jun2003.pdf
% For comparison, here are the existing overlap macros:
% \def\llap#1{\hbox to 0pt{\hss#1}}
% \def\rlap#1{\hbox to 0pt{#1\hss}}
\def\clap#1{\hbox to 0pt{\hss#1\hss}}
\def\mathllap{\mathpalette\mathllapinternal}
\def\mathrlap{\mathpalette\mathrlapinternal}
\def\mathclap{\mathpalette\mathclapinternal}
\def\mathllapinternal#1#2{%
\llap{$\mathsurround=0pt#1{#2}$}}
\def\mathrlapinternal#1#2{%
\rlap{$\mathsurround=0pt#1{#2}$}}
\def\mathclapinternal#1#2{%
\clap{$\mathsurround=0pt#1{#2}$}}

\newcommand{\alert}[1]{\textbf{\textcolor{red}{#1}}}




% [ Code blocks ]
%\DefineShortVerb[commandchars=\\\{\}]{\|}
\DefineVerbatimEnvironment{Highlighting}{Verbatim}{commandchars=\\\{\}}
% Add ',fontsize=\small' for more characters per line
\newenvironment{Shaded}{}{}
\newcommand{\KeywordTok}[1]{\textcolor[rgb]{0.00,0.44,0.13}{\textbf{{#1}}}}
\newcommand{\DataTypeTok}[1]{\textcolor[rgb]{0.56,0.13,0.00}{{#1}}}
\newcommand{\DecValTok}[1]{\textcolor[rgb]{0.25,0.63,0.44}{{#1}}}
\newcommand{\BaseNTok}[1]{\textcolor[rgb]{0.25,0.63,0.44}{{#1}}}
\newcommand{\FloatTok}[1]{\textcolor[rgb]{0.25,0.63,0.44}{{#1}}}
\newcommand{\CharTok}[1]{\textcolor[rgb]{0.25,0.44,0.63}{{#1}}}
\newcommand{\StringTok}[1]{\textcolor[rgb]{0.25,0.44,0.63}{{#1}}}
\newcommand{\CommentTok}[1]{\textcolor[rgb]{0.38,0.63,0.69}{\textit{{#1}}}}
\newcommand{\OtherTok}[1]{\textcolor[rgb]{0.00,0.44,0.13}{{#1}}}
\newcommand{\AlertTok}[1]{\textcolor[rgb]{1.00,0.00,0.00}{\textbf{{#1}}}}
\newcommand{\FunctionTok}[1]{\textcolor[rgb]{0.02,0.16,0.49}{{#1}}}
\newcommand{\RegionMarkerTok}[1]{{#1}}
\newcommand{\ErrorTok}[1]{\textcolor[rgb]{1.00,0.00,0.00}{\textbf{{#1}}}}
\newcommand{\NormalTok}[1]{{#1}}



    \def\preinserted{}
    \begin{document}
}

\def\postcommands{%
    \singlespacing
    \phantomsection
%    \bibliographystyle{../bibliography/expanded}
   \bibliographystyle{elsarticle-harv}
   \bibliography{../bibliography/references}
    \enddocument
}

\def\autoinsert{%
    \ifx\preinserted\undefined
        \expandafter\precommands
    \else
        \expandafter\postcommands
    \fi
}

\ifx\master\undefined\expandafter\autoinsert\fi


