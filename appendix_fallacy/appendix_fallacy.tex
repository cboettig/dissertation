

\input{../settings/boilerplate}

\chapter{Code for Prosecutors Fallacy}

This code is written in the \texttt{R} language for statistical
computing.\\Population dynamics are simulated using the
\texttt{populationdynamics} package (Boettiger, 2012) for exact
simulations of discrete birth-death processes in continuous time using
the Gillespie algorithm (Gillespie, 1977). Early warning signals of
variance and autocorrelation, as well as the model-based estimate of
Boettiger \& Hastings, (2012) are estimated using the
\texttt{earlywarning} package (Boettiger, 2012). These packages can be
installed from Github using the \texttt{devtools} R package

\begin{Shaded}
\begin{Highlighting}[]
\KeywordTok{library}\NormalTok{(devtools)}
\KeywordTok{install_github}\NormalTok{(}\StringTok{"populationdynamics"}\NormalTok{, }\StringTok{"cboettig"}\NormalTok{)}
\KeywordTok{install_github}\NormalTok{(}\StringTok{"earlywarning"}\NormalTok{, }\StringTok{"cboettig"}\NormalTok{)}
\end{Highlighting}
\end{Shaded}
In the examples of this manuscript, the population dynamics are given by

\begin{align}
  \frac{dP(n,t)}{dt} &= b_{n-1} P(n-1,t) + d_{n+1}P(n+1,t) - (b_n+d_n) P(n,t)  \label{master}, \\
    b_n &= \frac{e K n^2}{n^2 + h^2}, \\
    d_n &= e n + a,
\end{align}

which is provided by the \texttt{saddle\_node\_ibm} model in
\texttt{populationdynamics}.

For each of the warning signal statistics in question, we need to
generate the distibution over all replicates and then over replicates
which have been selected conditional on having experienced a crash.

We begin by loading the required libraries.

\begin{Shaded}
\begin{Highlighting}[]
\KeywordTok{library}\NormalTok{(populationdynamics)}
\KeywordTok{library}\NormalTok{(earlywarning)}
\KeywordTok{library}\NormalTok{(reshape2)       }\CommentTok{# data manipulation}
\KeywordTok{library}\NormalTok{(data.table) }\CommentTok{# data manipulation}
\KeywordTok{library}\NormalTok{(ggplot2)        }\CommentTok{# graphics}
\KeywordTok{library}\NormalTok{(snowfall)       }\CommentTok{# parallel}
\end{Highlighting}
\end{Shaded}
\subsubsection{Conditional distribution}

Then we fix a set of paramaters we will use for the simulation function.
Since we will simulate 20,000 replicates with 50,000 pts a piece, we can
save memory by performing the conditional selection on the ones that
crash as we go along and disgard the others. (We will create a null
distribution in which we ignore this conditional selection later).

\begin{Shaded}
\begin{Highlighting}[]
\NormalTok{select_crashes <- function(n)\{}
    \NormalTok{T<- }\DecValTok{5000}
    \NormalTok{n_pts <- n}
    \NormalTok{pars = }\KeywordTok{c}\NormalTok{(}\DataTypeTok{Xo =} \DecValTok{500}\NormalTok{, }\DataTypeTok{e =} \FloatTok{0.5}\NormalTok{, }\DataTypeTok{a =} \DecValTok{180}\NormalTok{, }\DataTypeTok{K =} \DecValTok{1000}\NormalTok{, }\DataTypeTok{h =} \DecValTok{200}\NormalTok{,}
    \DataTypeTok{i =} \DecValTok{0}\NormalTok{, }\DataTypeTok{Da =} \DecValTok{0}\NormalTok{, }\DataTypeTok{Dt =} \DecValTok{0}\NormalTok{, }\DataTypeTok{p =} \DecValTok{2}\NormalTok{)}
    \NormalTok{sn <- }\KeywordTok{saddle_node_ibm}\NormalTok{(pars, }\DataTypeTok{times=}\KeywordTok{seq}\NormalTok{(}\DecValTok{0}\NormalTok{,T, }\DataTypeTok{length=}\NormalTok{n_pts), }\DataTypeTok{reps=}\DecValTok{1000}\NormalTok{)}
    \NormalTok{d <- }\KeywordTok{dim}\NormalTok{(sn$x1)}
    \NormalTok{crashed <- }\KeywordTok{which}\NormalTok{(sn$x1[d[}\DecValTok{1}\NormalTok{],]==}\DecValTok{0}\NormalTok{)}
    \NormalTok{sn$x1[,crashed] }
\NormalTok{\}}
\end{Highlighting}
\end{Shaded}
To take advantage of parallelization, we loop over this function a set
number of times. The \texttt{snowfall} library provides the
parallelization of the \texttt{lapply} loop. We initialize a parallel
framework across 12 processors,

\begin{Shaded}
\begin{Highlighting}[]
\KeywordTok{sfInit}\NormalTok{(}\DataTypeTok{parallel=}\OtherTok{TRUE}\NormalTok{, }\DataTypeTok{cpu=}\DecValTok{12}\NormalTok{)}
\KeywordTok{sfLibrary}\NormalTok{(populationdynamics)}
\KeywordTok{sfExportAll}\NormalTok{()}
\end{Highlighting}
\end{Shaded}
and then loop over 20 instances of 1000 replicates each to assemble our
dataset. A few extra commands format the data into a table with columns
of times, replicate id number, and population value at the given time.

\begin{Shaded}
\begin{Highlighting}[]
\NormalTok{examples <- }\KeywordTok{sfLapply}\NormalTok{(}\DecValTok{1}\NormalTok{:}\DecValTok{20}\NormalTok{, function(i) }\KeywordTok{select_crashes}\NormalTok{(}\DecValTok{50000}\NormalTok{))}
\NormalTok{dat <- }\KeywordTok{melt}\NormalTok{(}\KeywordTok{as.matrix}\NormalTok{(}\KeywordTok{as.data.frame}\NormalTok{(examples, }\DataTypeTok{check.names=}\OtherTok{FALSE}\NormalTok{)))}
\KeywordTok{names}\NormalTok{(dat) = }\KeywordTok{c}\NormalTok{(}\StringTok{"time"}\NormalTok{, }\StringTok{"reps"}\NormalTok{, }\StringTok{"value"}\NormalTok{)}
\KeywordTok{levels}\NormalTok{(dat$reps) <- }\DecValTok{1}\NormalTok{:}\KeywordTok{length}\NormalTok{(}\KeywordTok{levels}\NormalTok{(dat$reps)) }\CommentTok{# use numbers for reps}
\end{Highlighting}
\end{Shaded}
Zoom in on the relevant area of data near the crash

\begin{Shaded}
\begin{Highlighting}[]
\KeywordTok{require}\NormalTok{(plyr)}
\NormalTok{zoom <- }\KeywordTok{ddply}\NormalTok{(dat, }\StringTok{"reps"}\NormalTok{, function(X)\{}
    \NormalTok{tip <- }\KeywordTok{min}\NormalTok{(}\KeywordTok{which}\NormalTok{(X$value==}\DecValTok{0}\NormalTok{))}
    \NormalTok{index <- }\KeywordTok{max}\NormalTok{(tip}\DecValTok{-500}\NormalTok{,}\DecValTok{1}\NormalTok{):tip}
    \KeywordTok{data.frame}\NormalTok{(}\DataTypeTok{time=}\NormalTok{X$time[index], }\DataTypeTok{value=}\NormalTok{X$value[index])}
    \NormalTok{\})}
\end{Highlighting}
\end{Shaded}
Finally we compute model-based warning signals on all each of these.\\We
take advantage of the \texttt{data.table} package to quickly apply the
\texttt{warningtrend} function over each of the replicates.

\begin{Shaded}
\begin{Highlighting}[]
\NormalTok{dt <- }\KeywordTok{data.table}\NormalTok{(}\KeywordTok{subset}\NormalTok{(zoom, value>}\DecValTok{250}\NormalTok{))}
\NormalTok{var <- dt[, }\KeywordTok{warningtrend}\NormalTok{(}
            \KeywordTok{data.frame}\NormalTok{(}\DataTypeTok{time=}\NormalTok{time, }\DataTypeTok{value=}\NormalTok{value), window_var),}
            \NormalTok{by=reps]$V1}
\NormalTok{acor <- dt[, }\KeywordTok{warningtrend}\NormalTok{(}\KeywordTok{data.frame}\NormalTok{(}\DataTypeTok{time=}\NormalTok{time, }\DataTypeTok{value=}\NormalTok{value),}
             \NormalTok{window_autocorr),}
             \NormalTok{by=reps]$V1}
\NormalTok{dat <- }\KeywordTok{melt}\NormalTok{(}\KeywordTok{data.frame}\NormalTok{(}\DataTypeTok{Variance=}\NormalTok{var, }\DataTypeTok{Autocorrelation=}\NormalTok{acor))}
\end{Highlighting}
\end{Shaded}
\subsubsection{Null distribution}

To compare against the expected distribution of these statistics, we
create another set of simulations without conditioning on having
experienced a chance transition, on which we perform the identical
analysis.

\begin{Shaded}
\begin{Highlighting}[]
\NormalTok{select_crashes <- function(n)\{}
    \NormalTok{T<- }\DecValTok{5000}
    \NormalTok{n_pts <- n}
    \NormalTok{pars = }\KeywordTok{c}\NormalTok{(}\DataTypeTok{Xo =} \DecValTok{500}\NormalTok{, }\DataTypeTok{e =} \FloatTok{0.5}\NormalTok{, }\DataTypeTok{a =} \DecValTok{180}\NormalTok{, }\DataTypeTok{K =} \DecValTok{1000}\NormalTok{, }\DataTypeTok{h =} \DecValTok{200}\NormalTok{,}
    \DataTypeTok{i =} \DecValTok{0}\NormalTok{, }\DataTypeTok{Da =} \DecValTok{0}\NormalTok{, }\DataTypeTok{Dt =} \DecValTok{0}\NormalTok{, }\DataTypeTok{p =} \DecValTok{2}\NormalTok{)}
    \NormalTok{sn <- }\KeywordTok{saddle_node_ibm}\NormalTok{(pars, }\DataTypeTok{times=}\KeywordTok{seq}\NormalTok{(}\DecValTok{0}\NormalTok{,T, }\DataTypeTok{length=}\NormalTok{n_pts), }\DataTypeTok{reps=}\DecValTok{500}\NormalTok{)}
    \NormalTok{d <- }\KeywordTok{dim}\NormalTok{(sn$x1)}
    \NormalTok{sn$x1[}\DecValTok{1}\NormalTok{:}\DecValTok{501}\NormalTok{,]}
\NormalTok{\}}
\end{Highlighting}
\end{Shaded}
\begin{Shaded}
\begin{Highlighting}[]
\KeywordTok{sfExportAll}\NormalTok{()}
\NormalTok{examples <-  }\KeywordTok{sfLapply}\NormalTok{(}\DecValTok{1}\NormalTok{:}\DecValTok{24}\NormalTok{, function(i) }\KeywordTok{select_crashes}\NormalTok{(}\DecValTok{50000}\NormalTok{))}
\NormalTok{nulldat <- }\KeywordTok{melt}\NormalTok{(}\KeywordTok{as.matrix}\NormalTok{(}\KeywordTok{as.data.frame}\NormalTok{(examples, }\DataTypeTok{check.names=}\OtherTok{FALSE}\NormalTok{)))}
\NormalTok{nulldat <- }\KeywordTok{melt}\NormalTok{(examples)}
\KeywordTok{names}\NormalTok{(nulldat) = }\KeywordTok{c}\NormalTok{(}\StringTok{"time"}\NormalTok{, }\StringTok{"reps"}\NormalTok{, }\StringTok{"value"}\NormalTok{)}
\KeywordTok{levels}\NormalTok{(nulldat$reps) <- }\DecValTok{1}\NormalTok{:}\KeywordTok{length}\NormalTok{(}\KeywordTok{levels}\NormalTok{(dat$reps)) }
\end{Highlighting}
\end{Shaded}
\begin{Shaded}
\begin{Highlighting}[]
\KeywordTok{require}\NormalTok{(plyr)}
\NormalTok{nullzoom <- }\KeywordTok{ddply}\NormalTok{(nulldat, }\StringTok{"reps"}\NormalTok{, function(X)\{}
    \KeywordTok{data.frame}\NormalTok{(}\DataTypeTok{time=}\NormalTok{X$time, }\DataTypeTok{value=}\NormalTok{X$value)}
    \NormalTok{\})}
\end{Highlighting}
\end{Shaded}
\begin{Shaded}
\begin{Highlighting}[]
\NormalTok{nulldt <- }\KeywordTok{data.table}\NormalTok{(nullzoom)}
\NormalTok{nullvar <- nulldt[, }\KeywordTok{warningtrend}\NormalTok{(}
                    \KeywordTok{data.frame}\NormalTok{(}\DataTypeTok{time=}\NormalTok{time, }\DataTypeTok{value=}\NormalTok{value), window_var),}
                    \NormalTok{by=reps]$V1}
\NormalTok{nullacor <- nulldt[, }\KeywordTok{warningtrend}\NormalTok{(}
                     \KeywordTok{data.frame}\NormalTok{(}\DataTypeTok{time=}\NormalTok{time, }\DataTypeTok{value=}\NormalTok{value), window_autocorr),}
                     \NormalTok{by=reps]$V1}
\NormalTok{nulldat <- }\KeywordTok{melt}\NormalTok{(}\KeywordTok{data.frame}\NormalTok{(}\DataTypeTok{Variance=}\NormalTok{nullvar, }\DataTypeTok{Autocorrelation=}\NormalTok{nullacor))}
\end{Highlighting}
\end{Shaded}
\subsection{Plot the distributions}

We generate the plot of the null distribution as a density curve
overlaid on the histogram of warning signal statistics we calculated for
the conditionally selected examples.

\begin{Shaded}
\begin{Highlighting}[]
\KeywordTok{ggplot}\NormalTok{(dat) + }
 \KeywordTok{geom_histogram}\NormalTok{(}\KeywordTok{aes}\NormalTok{(value, }\DataTypeTok{y=}\NormalTok{..density..), }\DataTypeTok{binwidth=}\FloatTok{0.2}\NormalTok{, }\DataTypeTok{alpha=}\NormalTok{.}\DecValTok{5}\NormalTok{) +}
 \KeywordTok{facet_wrap}\NormalTok{(~variable) + }\KeywordTok{xlim}\NormalTok{(}\KeywordTok{c}\NormalTok{(-}\DecValTok{1}\NormalTok{, }\DecValTok{1}\NormalTok{)) + }
 \KeywordTok{geom_density}\NormalTok{(}\DataTypeTok{data=}\NormalTok{nulldat, }\KeywordTok{aes}\NormalTok{(value), }\DataTypeTok{bw=}\FloatTok{0.2}\NormalTok{)}
\end{Highlighting}
\end{Shaded}
\begin{figure}[htbp]
\centering
\includegraphics[width=.5\linewidth]{../appendix_fallacy/figure2.png}
\caption{Figure2}
\end{figure}


\input{../settings/boilerplate}

