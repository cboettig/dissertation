
% BOILERPLATE - Allows for individual chapters to be compiled.
%
% Usage:
%    % BOILERPLATE - Allows for individual chapters to be compiled.
%
% Usage:
%    % BOILERPLATE - Allows for individual chapters to be compiled.
%
% Usage:
%    \input{../settings/boilerplate} % place at start and end of chapter
% 

\def\precommands{%
    \input{../settings/phdsetup}
    \def\preinserted{}
    \begin{document}
}

\def\postcommands{%
    \singlespacing
    \phantomsection
%    \bibliographystyle{../bibliography/expanded}
   \bibliographystyle{elsarticle-harv}
   \bibliography{../bibliography/references}
    \enddocument
}

\def\autoinsert{%
    \ifx\preinserted\undefined
        \expandafter\precommands
    \else
        \expandafter\postcommands
    \fi
}

\ifx\master\undefined\expandafter\autoinsert\fi

 % place at start and end of chapter
% 

\def\precommands{%
    % [ USER VARIABLES ]

\def\PHDTITLE {Regime shifts in ecology and evolution}
\def\PHDAUTHOR{Carl Boettiger}
\def\PHDSCHOOL{University of California, Davis}

\def\PHDMONTH {September}
\def\PHDYEAR  {2012}
\def\PHDDEPT {Center for Population Biology}

\def\BSSCHOOL {Princeton}
\def\BSYEAR   {2007}

\def\PHDCOMMITTEEA{Alan Hastings}
\def\PHDCOMMITTEEB{Peter Wainwright}
\def\PHDCOMMITTEEC{Brian Moore}

% [ GLOBAL SETUP ]

\documentclass[letterpaper,oneside,11pt]{report}


% [ CARL BOETTIGER`S CUSTOM COMMANDS, LIBRARIES, ETC ]

\usepackage{subfigure}
\usepackage[sort&compress]{natbib}
\usepackage{color}
\usepackage{fancyvrb}
\usepackage{ctable}


\usepackage{silence}
\WarningFilter{amsmath}{Underfull}     

%\newcommand{\argmax}{\operatorname{argmax}}
\newcommand{\ud}{\mathrm{d}}


\usepackage{calc}
\usepackage{breakcites}
\usepackage[newcommands]{ragged2e}
\usepackage{appendix}
\usepackage{comment}
\usepackage{xifthen}

\usepackage{graphicx}
\usepackage{epstopdf}


\renewenvironment{abstract}{\chapter*{Abstract}}{}
\renewcommand{\bibname}{Bibliography}
\renewcommand{\contentsname}{Table of Contents}

\makeatletter
\renewcommand{\@biblabel}[1]{\textsc{#1}}
\makeatother

% [ FONT SETTINGS ]

\usepackage[T1]{fontenc}
\usepackage{libertine}

\usepackage[tbtags, intlimits, namelimits]{amsmath}
\usepackage{amsthm}
\usepackage{amssymb}
\usepackage{amsfonts}



% [ PAGE LAYOUT ]

\usepackage{geometry}
\geometry{lmargin = 1.5in}
\geometry{rmargin = 1.0in}
\geometry{tmargin = 1.0in}
\geometry{bmargin = 1.0in}

% [ PDF SETTINGS ]

\usepackage[final]{hyperref}
\hypersetup{
    breaklinks  = {true},
    colorlinks  = {true},
    linktocpage = {false},
    linkcolor   = {blue},
    citecolor   = {black},
    urlcolor    = {black},
    plainpages  = {false},
    pageanchor  = {true},
    pdfauthor   = {\PHDAUTHOR},
    pdftitle    = {\PHDTITLE},
    pdfsubject  = {Dissertation, \PHDSCHOOL},
    pdfcreator  = {},
    pdfkeywords = {},
    pdfproducer = {}
}
\urlstyle{same}

% [ LETTER SPACING ]

\usepackage[final]{microtype}
\microtypesetup{protrusion=compatibility}
\microtypesetup{expansion=false}

\newcommand{\upper}[1]{\MakeUppercase{#1}}
\let\lsscshape\scshape

\ifcase\pdfoutput\else\microtypesetup{letterspace=15}
\renewcommand{\scshape}{\lsscshape\lsstyle}
\renewcommand{\upper}[1]{\textls[50]{\MakeUppercase{#1}}}\fi

% [ LINE SPACING ]

\usepackage[doublespacing]{setspace}
\renewcommand{\displayskipstretch}{0.75}

\setlength{\parskip   }{0em}
\setlength{\parindent }{2em}

% [ TABLE FORMATTING ]

\usepackage{booktabs}
\usepackage{multirow}
\usepackage{dcolumn}
\setlength{\heavyrulewidth}{1.5\arrayrulewidth}
\setlength{\lightrulewidth}{1.0\arrayrulewidth}
\setlength{\doublerulesep }{2.0\arrayrulewidth}

% [ SECTION FORMATTING ]

\usepackage[largestsep,nobottomtitles*]{titlesec}
\renewcommand{\bottomtitlespace}{0.75in}

\titleformat{\chapter}[display]%
    {\bfseries\huge\singlespacing}%
    {\filleft\textsc{\LARGE \chaptertitlename\ \thechapter}}%
    {-0.2ex}{\titlerule[3pt]\vspace{0.2ex}}[]

\titleformat{\section}{\LARGE}{\thesection\hspace{0.5em}}{0ex}{}
\titleformat{\subsection}{\Large}{\thesubsection\hspace{0.5em}}{0ex}{}
\titleformat{\subsubsection}{\large}{\thesubsubsection\hspace{0.5em}}{0ex}{}

\titlespacing*{\chapter}{0em}{6ex}{4ex plus 2ex minus 0ex}
\titlespacing*{\section}{0em}{2ex plus 3ex minus 1ex}{0.5ex plus 0.5ex minus 0.5ex}
\titlespacing*{\subsection}{0ex}{2ex plus 3ex minus 1ex}{0ex}
\titlespacing*{\subsubsection}{0ex}{2ex plus 0ex minus 1ex}{0ex}

% [ HEADER SETTINGS ]

\usepackage{fancyhdr}

\setlength{\headheight}{\normalbaselineskip}
\setlength{\footskip  }{0.5in}
\setlength{\headsep   }{0.5in-\headheight}

\fancyheadoffset[R]{0.5in}
\renewcommand{\headrulewidth}{0pt}
\renewcommand{\footrulewidth}{0pt}

\newcommand{\pagebox}{\parbox[r][\headheight][t]{0.5in}{\hspace\fill\thepage}}

\newcommand{\prelimheaders}{\ifx\prelim\undefined\renewcommand{\thepage}{\textit{\roman{page}}}\fancypagestyle{plain}{\fancyhf{}\fancyfoot[L]{\makebox[\textwidth-0.5in]{\thepage}}}\pagestyle{plain}\def\prelim{}\fi}

\newcommand{\normalheaders}{\renewcommand{\thepage}{\arabic{page}}\fancypagestyle{plain}{\fancyhf{}\fancyhead[R]{\pagebox}}\pagestyle{plain}}

\normalheaders{}

% [ CUSTOM COMMANDS ]

\newcommand{\signaturebox}[1]{\multicolumn{1}{p{4in}}{\vspace{3ex}}\\\midrule #1\\}

% Redefine AMS proof environment to have itshape
% Note: This environment automatically adds \qed at the end. If your proof
% ends in a math environment, the \qed is placed, undesirably, on a new line.
% To prevent that, insert \qedhere inside the math environment.
\makeatletter
\renewenvironment{proof}[1][\proofname]{%
\par\pushQED{\qed}\normalfont%
\topsep6\p@\@plus6\p@\relax\trivlist%
\item[\hskip\labelsep\bfseries#1\@addpunct{.}]\itshape\ignorespaces}{%
\popQED\endtrivlist\@endpefalse}%
\makeatother

% TUGboat, Volume 0 (2001), No. 0
% http://math.arizona.edu/~aprl/publications/mathclap/perlis_mathclap_24Jun2003.pdf
% For comparison, here are the existing overlap macros:
% \def\llap#1{\hbox to 0pt{\hss#1}}
% \def\rlap#1{\hbox to 0pt{#1\hss}}
\def\clap#1{\hbox to 0pt{\hss#1\hss}}
\def\mathllap{\mathpalette\mathllapinternal}
\def\mathrlap{\mathpalette\mathrlapinternal}
\def\mathclap{\mathpalette\mathclapinternal}
\def\mathllapinternal#1#2{%
\llap{$\mathsurround=0pt#1{#2}$}}
\def\mathrlapinternal#1#2{%
\rlap{$\mathsurround=0pt#1{#2}$}}
\def\mathclapinternal#1#2{%
\clap{$\mathsurround=0pt#1{#2}$}}

\newcommand{\alert}[1]{\textbf{\textcolor{red}{#1}}}




% [ Code blocks ]
%\DefineShortVerb[commandchars=\\\{\}]{\|}
\DefineVerbatimEnvironment{Highlighting}{Verbatim}{commandchars=\\\{\}}
% Add ',fontsize=\small' for more characters per line
\newenvironment{Shaded}{}{}
\newcommand{\KeywordTok}[1]{\textcolor[rgb]{0.00,0.44,0.13}{\textbf{{#1}}}}
\newcommand{\DataTypeTok}[1]{\textcolor[rgb]{0.56,0.13,0.00}{{#1}}}
\newcommand{\DecValTok}[1]{\textcolor[rgb]{0.25,0.63,0.44}{{#1}}}
\newcommand{\BaseNTok}[1]{\textcolor[rgb]{0.25,0.63,0.44}{{#1}}}
\newcommand{\FloatTok}[1]{\textcolor[rgb]{0.25,0.63,0.44}{{#1}}}
\newcommand{\CharTok}[1]{\textcolor[rgb]{0.25,0.44,0.63}{{#1}}}
\newcommand{\StringTok}[1]{\textcolor[rgb]{0.25,0.44,0.63}{{#1}}}
\newcommand{\CommentTok}[1]{\textcolor[rgb]{0.38,0.63,0.69}{\textit{{#1}}}}
\newcommand{\OtherTok}[1]{\textcolor[rgb]{0.00,0.44,0.13}{{#1}}}
\newcommand{\AlertTok}[1]{\textcolor[rgb]{1.00,0.00,0.00}{\textbf{{#1}}}}
\newcommand{\FunctionTok}[1]{\textcolor[rgb]{0.02,0.16,0.49}{{#1}}}
\newcommand{\RegionMarkerTok}[1]{{#1}}
\newcommand{\ErrorTok}[1]{\textcolor[rgb]{1.00,0.00,0.00}{\textbf{{#1}}}}
\newcommand{\NormalTok}[1]{{#1}}



    \def\preinserted{}
    \begin{document}
}

\def\postcommands{%
    \singlespacing
    \phantomsection
%    \bibliographystyle{../bibliography/expanded}
   \bibliographystyle{elsarticle-harv}
   \bibliography{../bibliography/references}
    \enddocument
}

\def\autoinsert{%
    \ifx\preinserted\undefined
        \expandafter\precommands
    \else
        \expandafter\postcommands
    \fi
}

\ifx\master\undefined\expandafter\autoinsert\fi

 % place at start and end of chapter
% 

\def\precommands{%
    % [ USER VARIABLES ]

\def\PHDTITLE {Regime shifts in ecology and evolution}
\def\PHDAUTHOR{Carl Boettiger}
\def\PHDSCHOOL{University of California, Davis}

\def\PHDMONTH {September}
\def\PHDYEAR  {2012}
\def\PHDDEPT {Center for Population Biology}

\def\BSSCHOOL {Princeton}
\def\BSYEAR   {2007}

\def\PHDCOMMITTEEA{Alan Hastings}
\def\PHDCOMMITTEEB{Peter Wainwright}
\def\PHDCOMMITTEEC{Brian Moore}

% [ GLOBAL SETUP ]

\documentclass[letterpaper,oneside,11pt]{report}


% [ CARL BOETTIGER`S CUSTOM COMMANDS, LIBRARIES, ETC ]

\usepackage{subfigure}
\usepackage[sort&compress]{natbib}
\usepackage{color}
\usepackage{fancyvrb}
\usepackage{ctable}


\usepackage{silence}
\WarningFilter{amsmath}{Underfull}     

%\newcommand{\argmax}{\operatorname{argmax}}
\newcommand{\ud}{\mathrm{d}}


\usepackage{calc}
\usepackage{breakcites}
\usepackage[newcommands]{ragged2e}
\usepackage{appendix}
\usepackage{comment}
\usepackage{xifthen}

\usepackage{graphicx}
\usepackage{epstopdf}


\renewenvironment{abstract}{\chapter*{Abstract}}{}
\renewcommand{\bibname}{Bibliography}
\renewcommand{\contentsname}{Table of Contents}

\makeatletter
\renewcommand{\@biblabel}[1]{\textsc{#1}}
\makeatother

% [ FONT SETTINGS ]

\usepackage[T1]{fontenc}
\usepackage{libertine}

\usepackage[tbtags, intlimits, namelimits]{amsmath}
\usepackage{amsthm}
\usepackage{amssymb}
\usepackage{amsfonts}



% [ PAGE LAYOUT ]

\usepackage{geometry}
\geometry{lmargin = 1.5in}
\geometry{rmargin = 1.0in}
\geometry{tmargin = 1.0in}
\geometry{bmargin = 1.0in}

% [ PDF SETTINGS ]

\usepackage[final]{hyperref}
\hypersetup{
    breaklinks  = {true},
    colorlinks  = {true},
    linktocpage = {false},
    linkcolor   = {blue},
    citecolor   = {black},
    urlcolor    = {black},
    plainpages  = {false},
    pageanchor  = {true},
    pdfauthor   = {\PHDAUTHOR},
    pdftitle    = {\PHDTITLE},
    pdfsubject  = {Dissertation, \PHDSCHOOL},
    pdfcreator  = {},
    pdfkeywords = {},
    pdfproducer = {}
}
\urlstyle{same}

% [ LETTER SPACING ]

\usepackage[final]{microtype}
\microtypesetup{protrusion=compatibility}
\microtypesetup{expansion=false}

\newcommand{\upper}[1]{\MakeUppercase{#1}}
\let\lsscshape\scshape

\ifcase\pdfoutput\else\microtypesetup{letterspace=15}
\renewcommand{\scshape}{\lsscshape\lsstyle}
\renewcommand{\upper}[1]{\textls[50]{\MakeUppercase{#1}}}\fi

% [ LINE SPACING ]

\usepackage[doublespacing]{setspace}
\renewcommand{\displayskipstretch}{0.75}

\setlength{\parskip   }{0em}
\setlength{\parindent }{2em}

% [ TABLE FORMATTING ]

\usepackage{booktabs}
\usepackage{multirow}
\usepackage{dcolumn}
\setlength{\heavyrulewidth}{1.5\arrayrulewidth}
\setlength{\lightrulewidth}{1.0\arrayrulewidth}
\setlength{\doublerulesep }{2.0\arrayrulewidth}

% [ SECTION FORMATTING ]

\usepackage[largestsep,nobottomtitles*]{titlesec}
\renewcommand{\bottomtitlespace}{0.75in}

\titleformat{\chapter}[display]%
    {\bfseries\huge\singlespacing}%
    {\filleft\textsc{\LARGE \chaptertitlename\ \thechapter}}%
    {-0.2ex}{\titlerule[3pt]\vspace{0.2ex}}[]

\titleformat{\section}{\LARGE}{\thesection\hspace{0.5em}}{0ex}{}
\titleformat{\subsection}{\Large}{\thesubsection\hspace{0.5em}}{0ex}{}
\titleformat{\subsubsection}{\large}{\thesubsubsection\hspace{0.5em}}{0ex}{}

\titlespacing*{\chapter}{0em}{6ex}{4ex plus 2ex minus 0ex}
\titlespacing*{\section}{0em}{2ex plus 3ex minus 1ex}{0.5ex plus 0.5ex minus 0.5ex}
\titlespacing*{\subsection}{0ex}{2ex plus 3ex minus 1ex}{0ex}
\titlespacing*{\subsubsection}{0ex}{2ex plus 0ex minus 1ex}{0ex}

% [ HEADER SETTINGS ]

\usepackage{fancyhdr}

\setlength{\headheight}{\normalbaselineskip}
\setlength{\footskip  }{0.5in}
\setlength{\headsep   }{0.5in-\headheight}

\fancyheadoffset[R]{0.5in}
\renewcommand{\headrulewidth}{0pt}
\renewcommand{\footrulewidth}{0pt}

\newcommand{\pagebox}{\parbox[r][\headheight][t]{0.5in}{\hspace\fill\thepage}}

\newcommand{\prelimheaders}{\ifx\prelim\undefined\renewcommand{\thepage}{\textit{\roman{page}}}\fancypagestyle{plain}{\fancyhf{}\fancyfoot[L]{\makebox[\textwidth-0.5in]{\thepage}}}\pagestyle{plain}\def\prelim{}\fi}

\newcommand{\normalheaders}{\renewcommand{\thepage}{\arabic{page}}\fancypagestyle{plain}{\fancyhf{}\fancyhead[R]{\pagebox}}\pagestyle{plain}}

\normalheaders{}

% [ CUSTOM COMMANDS ]

\newcommand{\signaturebox}[1]{\multicolumn{1}{p{4in}}{\vspace{3ex}}\\\midrule #1\\}

% Redefine AMS proof environment to have itshape
% Note: This environment automatically adds \qed at the end. If your proof
% ends in a math environment, the \qed is placed, undesirably, on a new line.
% To prevent that, insert \qedhere inside the math environment.
\makeatletter
\renewenvironment{proof}[1][\proofname]{%
\par\pushQED{\qed}\normalfont%
\topsep6\p@\@plus6\p@\relax\trivlist%
\item[\hskip\labelsep\bfseries#1\@addpunct{.}]\itshape\ignorespaces}{%
\popQED\endtrivlist\@endpefalse}%
\makeatother

% TUGboat, Volume 0 (2001), No. 0
% http://math.arizona.edu/~aprl/publications/mathclap/perlis_mathclap_24Jun2003.pdf
% For comparison, here are the existing overlap macros:
% \def\llap#1{\hbox to 0pt{\hss#1}}
% \def\rlap#1{\hbox to 0pt{#1\hss}}
\def\clap#1{\hbox to 0pt{\hss#1\hss}}
\def\mathllap{\mathpalette\mathllapinternal}
\def\mathrlap{\mathpalette\mathrlapinternal}
\def\mathclap{\mathpalette\mathclapinternal}
\def\mathllapinternal#1#2{%
\llap{$\mathsurround=0pt#1{#2}$}}
\def\mathrlapinternal#1#2{%
\rlap{$\mathsurround=0pt#1{#2}$}}
\def\mathclapinternal#1#2{%
\clap{$\mathsurround=0pt#1{#2}$}}

\newcommand{\alert}[1]{\textbf{\textcolor{red}{#1}}}




% [ Code blocks ]
%\DefineShortVerb[commandchars=\\\{\}]{\|}
\DefineVerbatimEnvironment{Highlighting}{Verbatim}{commandchars=\\\{\}}
% Add ',fontsize=\small' for more characters per line
\newenvironment{Shaded}{}{}
\newcommand{\KeywordTok}[1]{\textcolor[rgb]{0.00,0.44,0.13}{\textbf{{#1}}}}
\newcommand{\DataTypeTok}[1]{\textcolor[rgb]{0.56,0.13,0.00}{{#1}}}
\newcommand{\DecValTok}[1]{\textcolor[rgb]{0.25,0.63,0.44}{{#1}}}
\newcommand{\BaseNTok}[1]{\textcolor[rgb]{0.25,0.63,0.44}{{#1}}}
\newcommand{\FloatTok}[1]{\textcolor[rgb]{0.25,0.63,0.44}{{#1}}}
\newcommand{\CharTok}[1]{\textcolor[rgb]{0.25,0.44,0.63}{{#1}}}
\newcommand{\StringTok}[1]{\textcolor[rgb]{0.25,0.44,0.63}{{#1}}}
\newcommand{\CommentTok}[1]{\textcolor[rgb]{0.38,0.63,0.69}{\textit{{#1}}}}
\newcommand{\OtherTok}[1]{\textcolor[rgb]{0.00,0.44,0.13}{{#1}}}
\newcommand{\AlertTok}[1]{\textcolor[rgb]{1.00,0.00,0.00}{\textbf{{#1}}}}
\newcommand{\FunctionTok}[1]{\textcolor[rgb]{0.02,0.16,0.49}{{#1}}}
\newcommand{\RegionMarkerTok}[1]{{#1}}
\newcommand{\ErrorTok}[1]{\textcolor[rgb]{1.00,0.00,0.00}{\textbf{{#1}}}}
\newcommand{\NormalTok}[1]{{#1}}



    \def\preinserted{}
    \begin{document}
}

\def\postcommands{%
    \singlespacing
    \phantomsection
%    \bibliographystyle{../bibliography/expanded}
   \bibliographystyle{elsarticle-harv}
   \bibliography{../bibliography/references}
    \enddocument
}

\def\autoinsert{%
    \ifx\preinserted\undefined
        \expandafter\precommands
    \else
        \expandafter\postcommands
    \fi
}

\ifx\master\undefined\expandafter\autoinsert\fi


\chapter{treebase Appendix}
\section{Reproducible computation: A diversification rate analysis}

Different diversification models make different assumptions about the
rate of speciation, extinction, and how these rates may be changing over
time. The authors consider eight different models, implemented in the
laser package (Rabosky 2006). This code fits each of the eight models to
that data:

\begin{Shaded}
\begin{Highlighting}[]
\KeywordTok{library}\NormalTok{(ape)}
\KeywordTok{library}\NormalTok{(laser)}
\NormalTok{bt <- }\KeywordTok{branching.times}\NormalTok{(derryberry)}
\NormalTok{models <- }\KeywordTok{list}\NormalTok{(}
  \DataTypeTok{yule =} \KeywordTok{pureBirth}\NormalTok{(bt),  }
  \DataTypeTok{birth_death =} \KeywordTok{bd}\NormalTok{(bt),     }
  \DataTypeTok{yule.2.rate =} \KeywordTok{yule2rate}\NormalTok{(bt),}
  \DataTypeTok{linear.diversity.dependent =} \KeywordTok{DDL}\NormalTok{(bt),    }
  \DataTypeTok{exponential.diversity.dependent =} \KeywordTok{DDX}\NormalTok{(bt),}
  \DataTypeTok{varying.speciation_rate =} \KeywordTok{fitSPVAR}\NormalTok{(bt),  }
  \DataTypeTok{varying.extinction_rate =} \KeywordTok{fitEXVAR}\NormalTok{(bt),  }
  \DataTypeTok{varying_both =} \KeywordTok{fitBOTHVAR}\NormalTok{(bt))}
\end{Highlighting}
\end{Shaded}
Each of the model estimate includes an AIC score indicating the goodness
of fit, penalized by model complexity (lower scores indicate better
fits) We ask R to tell us which model has the lowest AIC score,

\begin{Shaded}
\begin{Highlighting}[]
\NormalTok{aics <- }\KeywordTok{sapply}\NormalTok{(models, function(model) model$aic)}
\NormalTok{best_fit <- }\KeywordTok{names}\NormalTok{(models[}\KeywordTok{which.min}\NormalTok{(aics)])}
\end{Highlighting}
\end{Shaded}
and confirm the result presented in E. P. Derryberry et al. (2011); that
the yule.2.rate model is the best fit to the data.

The best-fit model in the laser analysis was a Yule (net diversification
rate) model with two separate rates. We can ask \texttt{TreePar} to see
if a model with more rate shifts is favoured over this single shift, a
question that was not possible to address using the tools provided in
\texttt{laser}. The previous analysis also considers a birth-death model
that allowed speciation and extinction rates to be estimated separately,
but did not allow for a shift in the rate of such a model. In the main
text we introduced a model from Stadler (2011) that permitted up to 3
change-points in the speciation rate of the Yule model,

\begin{Shaded}
\begin{Highlighting}[]
\NormalTok{yule_models <- }\KeywordTok{bd.shifts.optim}\NormalTok{(x, }\DataTypeTok{sampling =} \KeywordTok{c}\NormalTok{(}\DecValTok{1}\NormalTok{,}\DecValTok{1}\NormalTok{,}\DecValTok{1}\NormalTok{,}\DecValTok{1}\NormalTok{), }
  \DataTypeTok{grid =} \DecValTok{5}\NormalTok{, }\DataTypeTok{start =} \DecValTok{0}\NormalTok{, }\DataTypeTok{end =} \DecValTok{60}\NormalTok{, }\DataTypeTok{yule =} \OtherTok{TRUE}\NormalTok{)[[}\DecValTok{2}\NormalTok{]]}
\end{Highlighting}
\end{Shaded}
We can also compare the performance of models which allow up to three
shifts while estimating extinction and speciation rates separately:

\begin{Shaded}
\begin{Highlighting}[]
\NormalTok{birth_death_models <- }\KeywordTok{bd.shifts.optim}\NormalTok{(x, }\DataTypeTok{sampling =} \KeywordTok{c}\NormalTok{(}\DecValTok{1}\NormalTok{,}\DecValTok{1}\NormalTok{,}\DecValTok{1}\NormalTok{,}\DecValTok{1}\NormalTok{), }
  \DataTypeTok{grid =} \DecValTok{5}\NormalTok{, }\DataTypeTok{start =} \DecValTok{0}\NormalTok{, }\DataTypeTok{end =} \DecValTok{60}\NormalTok{, }\DataTypeTok{yule =} \OtherTok{FALSE}\NormalTok{)[[}\DecValTok{2}\NormalTok{]]}
\end{Highlighting}
\end{Shaded}
The models output by these functions are ordered by increasing number of
shifts.\\We can select the best-fitting model by AIC score, which is
slightly cumbersome in \texttt{TreePar} syntax. First compute the AIC
scores of both the \texttt{yule\_models} and the
\texttt{birth\_death\_models} we fitted above,

\begin{Shaded}
\begin{Highlighting}[]
\NormalTok{yule_aic <- }
\KeywordTok{sapply}\NormalTok{(yule_models, function(pars)}
                    \DecValTok{2} \NormalTok{* (}\KeywordTok{length}\NormalTok{(pars) - }\DecValTok{1}\NormalTok{) + }\DecValTok{2} \NormalTok{* pars[}\DecValTok{1}\NormalTok{] )}
\NormalTok{birth_death_aic <- }
\KeywordTok{sapply}\NormalTok{(birth_death_models, function(pars)}
                            \DecValTok{2} \NormalTok{* (}\KeywordTok{length}\NormalTok{(pars) - }\DecValTok{1}\NormalTok{) + }\DecValTok{2} \NormalTok{* pars[}\DecValTok{1}\NormalTok{] )}
\end{Highlighting}
\end{Shaded}
And then generate a list identifying which model has the best (lowest)
AIC score among the Yule models and which has the best AIC score among
the birth-death models,

\begin{Shaded}
\begin{Highlighting}[]
\NormalTok{best_no_of_rates <- }\KeywordTok{list}\NormalTok{(}\DataTypeTok{Yule =} \KeywordTok{which.min}\NormalTok{(yule_aic), }
                         \DataTypeTok{birth.death =} \KeywordTok{which.min}\NormalTok{(birth_death_aic))}
\end{Highlighting}
\end{Shaded}
The best model is then whichever of these has the smaller AIC value.

\begin{Shaded}
\begin{Highlighting}[]
\NormalTok{best_model <- }\KeywordTok{which.min}\NormalTok{(}\KeywordTok{c}\NormalTok{(}\KeywordTok{min}\NormalTok{(yule_aic), }\KeywordTok{min}\NormalTok{(birth_death_aic)))}
\end{Highlighting}
\end{Shaded}
which confirms that the Yule 2-rate\\model is still the best choice
based on AIC score. Of the eight models in this second analysis, only
three were in the original set considered (Yule 1-rate and 2-rate, and
birth-death without a shift), so we could by no means have been sure
ahead of time that a birth death with a shift, or a Yule model with a
greater number of shifts, would not have fitted better.

% BOILERPLATE - Allows for individual chapters to be compiled.
%
% Usage:
%    % BOILERPLATE - Allows for individual chapters to be compiled.
%
% Usage:
%    % BOILERPLATE - Allows for individual chapters to be compiled.
%
% Usage:
%    \input{../settings/boilerplate} % place at start and end of chapter
% 

\def\precommands{%
    \input{../settings/phdsetup}
    \def\preinserted{}
    \begin{document}
}

\def\postcommands{%
    \singlespacing
    \phantomsection
%    \bibliographystyle{../bibliography/expanded}
   \bibliographystyle{elsarticle-harv}
   \bibliography{../bibliography/references}
    \enddocument
}

\def\autoinsert{%
    \ifx\preinserted\undefined
        \expandafter\precommands
    \else
        \expandafter\postcommands
    \fi
}

\ifx\master\undefined\expandafter\autoinsert\fi

 % place at start and end of chapter
% 

\def\precommands{%
    % [ USER VARIABLES ]

\def\PHDTITLE {Regime shifts in ecology and evolution}
\def\PHDAUTHOR{Carl Boettiger}
\def\PHDSCHOOL{University of California, Davis}

\def\PHDMONTH {September}
\def\PHDYEAR  {2012}
\def\PHDDEPT {Center for Population Biology}

\def\BSSCHOOL {Princeton}
\def\BSYEAR   {2007}

\def\PHDCOMMITTEEA{Alan Hastings}
\def\PHDCOMMITTEEB{Peter Wainwright}
\def\PHDCOMMITTEEC{Brian Moore}

% [ GLOBAL SETUP ]

\documentclass[letterpaper,oneside,11pt]{report}


% [ CARL BOETTIGER`S CUSTOM COMMANDS, LIBRARIES, ETC ]

\usepackage{subfigure}
\usepackage[sort&compress]{natbib}
\usepackage{color}
\usepackage{fancyvrb}
\usepackage{ctable}


\usepackage{silence}
\WarningFilter{amsmath}{Underfull}     

%\newcommand{\argmax}{\operatorname{argmax}}
\newcommand{\ud}{\mathrm{d}}


\usepackage{calc}
\usepackage{breakcites}
\usepackage[newcommands]{ragged2e}
\usepackage{appendix}
\usepackage{comment}
\usepackage{xifthen}

\usepackage{graphicx}
\usepackage{epstopdf}


\renewenvironment{abstract}{\chapter*{Abstract}}{}
\renewcommand{\bibname}{Bibliography}
\renewcommand{\contentsname}{Table of Contents}

\makeatletter
\renewcommand{\@biblabel}[1]{\textsc{#1}}
\makeatother

% [ FONT SETTINGS ]

\usepackage[T1]{fontenc}
\usepackage{libertine}

\usepackage[tbtags, intlimits, namelimits]{amsmath}
\usepackage{amsthm}
\usepackage{amssymb}
\usepackage{amsfonts}



% [ PAGE LAYOUT ]

\usepackage{geometry}
\geometry{lmargin = 1.5in}
\geometry{rmargin = 1.0in}
\geometry{tmargin = 1.0in}
\geometry{bmargin = 1.0in}

% [ PDF SETTINGS ]

\usepackage[final]{hyperref}
\hypersetup{
    breaklinks  = {true},
    colorlinks  = {true},
    linktocpage = {false},
    linkcolor   = {blue},
    citecolor   = {black},
    urlcolor    = {black},
    plainpages  = {false},
    pageanchor  = {true},
    pdfauthor   = {\PHDAUTHOR},
    pdftitle    = {\PHDTITLE},
    pdfsubject  = {Dissertation, \PHDSCHOOL},
    pdfcreator  = {},
    pdfkeywords = {},
    pdfproducer = {}
}
\urlstyle{same}

% [ LETTER SPACING ]

\usepackage[final]{microtype}
\microtypesetup{protrusion=compatibility}
\microtypesetup{expansion=false}

\newcommand{\upper}[1]{\MakeUppercase{#1}}
\let\lsscshape\scshape

\ifcase\pdfoutput\else\microtypesetup{letterspace=15}
\renewcommand{\scshape}{\lsscshape\lsstyle}
\renewcommand{\upper}[1]{\textls[50]{\MakeUppercase{#1}}}\fi

% [ LINE SPACING ]

\usepackage[doublespacing]{setspace}
\renewcommand{\displayskipstretch}{0.75}

\setlength{\parskip   }{0em}
\setlength{\parindent }{2em}

% [ TABLE FORMATTING ]

\usepackage{booktabs}
\usepackage{multirow}
\usepackage{dcolumn}
\setlength{\heavyrulewidth}{1.5\arrayrulewidth}
\setlength{\lightrulewidth}{1.0\arrayrulewidth}
\setlength{\doublerulesep }{2.0\arrayrulewidth}

% [ SECTION FORMATTING ]

\usepackage[largestsep,nobottomtitles*]{titlesec}
\renewcommand{\bottomtitlespace}{0.75in}

\titleformat{\chapter}[display]%
    {\bfseries\huge\singlespacing}%
    {\filleft\textsc{\LARGE \chaptertitlename\ \thechapter}}%
    {-0.2ex}{\titlerule[3pt]\vspace{0.2ex}}[]

\titleformat{\section}{\LARGE}{\thesection\hspace{0.5em}}{0ex}{}
\titleformat{\subsection}{\Large}{\thesubsection\hspace{0.5em}}{0ex}{}
\titleformat{\subsubsection}{\large}{\thesubsubsection\hspace{0.5em}}{0ex}{}

\titlespacing*{\chapter}{0em}{6ex}{4ex plus 2ex minus 0ex}
\titlespacing*{\section}{0em}{2ex plus 3ex minus 1ex}{0.5ex plus 0.5ex minus 0.5ex}
\titlespacing*{\subsection}{0ex}{2ex plus 3ex minus 1ex}{0ex}
\titlespacing*{\subsubsection}{0ex}{2ex plus 0ex minus 1ex}{0ex}

% [ HEADER SETTINGS ]

\usepackage{fancyhdr}

\setlength{\headheight}{\normalbaselineskip}
\setlength{\footskip  }{0.5in}
\setlength{\headsep   }{0.5in-\headheight}

\fancyheadoffset[R]{0.5in}
\renewcommand{\headrulewidth}{0pt}
\renewcommand{\footrulewidth}{0pt}

\newcommand{\pagebox}{\parbox[r][\headheight][t]{0.5in}{\hspace\fill\thepage}}

\newcommand{\prelimheaders}{\ifx\prelim\undefined\renewcommand{\thepage}{\textit{\roman{page}}}\fancypagestyle{plain}{\fancyhf{}\fancyfoot[L]{\makebox[\textwidth-0.5in]{\thepage}}}\pagestyle{plain}\def\prelim{}\fi}

\newcommand{\normalheaders}{\renewcommand{\thepage}{\arabic{page}}\fancypagestyle{plain}{\fancyhf{}\fancyhead[R]{\pagebox}}\pagestyle{plain}}

\normalheaders{}

% [ CUSTOM COMMANDS ]

\newcommand{\signaturebox}[1]{\multicolumn{1}{p{4in}}{\vspace{3ex}}\\\midrule #1\\}

% Redefine AMS proof environment to have itshape
% Note: This environment automatically adds \qed at the end. If your proof
% ends in a math environment, the \qed is placed, undesirably, on a new line.
% To prevent that, insert \qedhere inside the math environment.
\makeatletter
\renewenvironment{proof}[1][\proofname]{%
\par\pushQED{\qed}\normalfont%
\topsep6\p@\@plus6\p@\relax\trivlist%
\item[\hskip\labelsep\bfseries#1\@addpunct{.}]\itshape\ignorespaces}{%
\popQED\endtrivlist\@endpefalse}%
\makeatother

% TUGboat, Volume 0 (2001), No. 0
% http://math.arizona.edu/~aprl/publications/mathclap/perlis_mathclap_24Jun2003.pdf
% For comparison, here are the existing overlap macros:
% \def\llap#1{\hbox to 0pt{\hss#1}}
% \def\rlap#1{\hbox to 0pt{#1\hss}}
\def\clap#1{\hbox to 0pt{\hss#1\hss}}
\def\mathllap{\mathpalette\mathllapinternal}
\def\mathrlap{\mathpalette\mathrlapinternal}
\def\mathclap{\mathpalette\mathclapinternal}
\def\mathllapinternal#1#2{%
\llap{$\mathsurround=0pt#1{#2}$}}
\def\mathrlapinternal#1#2{%
\rlap{$\mathsurround=0pt#1{#2}$}}
\def\mathclapinternal#1#2{%
\clap{$\mathsurround=0pt#1{#2}$}}

\newcommand{\alert}[1]{\textbf{\textcolor{red}{#1}}}




% [ Code blocks ]
%\DefineShortVerb[commandchars=\\\{\}]{\|}
\DefineVerbatimEnvironment{Highlighting}{Verbatim}{commandchars=\\\{\}}
% Add ',fontsize=\small' for more characters per line
\newenvironment{Shaded}{}{}
\newcommand{\KeywordTok}[1]{\textcolor[rgb]{0.00,0.44,0.13}{\textbf{{#1}}}}
\newcommand{\DataTypeTok}[1]{\textcolor[rgb]{0.56,0.13,0.00}{{#1}}}
\newcommand{\DecValTok}[1]{\textcolor[rgb]{0.25,0.63,0.44}{{#1}}}
\newcommand{\BaseNTok}[1]{\textcolor[rgb]{0.25,0.63,0.44}{{#1}}}
\newcommand{\FloatTok}[1]{\textcolor[rgb]{0.25,0.63,0.44}{{#1}}}
\newcommand{\CharTok}[1]{\textcolor[rgb]{0.25,0.44,0.63}{{#1}}}
\newcommand{\StringTok}[1]{\textcolor[rgb]{0.25,0.44,0.63}{{#1}}}
\newcommand{\CommentTok}[1]{\textcolor[rgb]{0.38,0.63,0.69}{\textit{{#1}}}}
\newcommand{\OtherTok}[1]{\textcolor[rgb]{0.00,0.44,0.13}{{#1}}}
\newcommand{\AlertTok}[1]{\textcolor[rgb]{1.00,0.00,0.00}{\textbf{{#1}}}}
\newcommand{\FunctionTok}[1]{\textcolor[rgb]{0.02,0.16,0.49}{{#1}}}
\newcommand{\RegionMarkerTok}[1]{{#1}}
\newcommand{\ErrorTok}[1]{\textcolor[rgb]{1.00,0.00,0.00}{\textbf{{#1}}}}
\newcommand{\NormalTok}[1]{{#1}}



    \def\preinserted{}
    \begin{document}
}

\def\postcommands{%
    \singlespacing
    \phantomsection
%    \bibliographystyle{../bibliography/expanded}
   \bibliographystyle{elsarticle-harv}
   \bibliography{../bibliography/references}
    \enddocument
}

\def\autoinsert{%
    \ifx\preinserted\undefined
        \expandafter\precommands
    \else
        \expandafter\postcommands
    \fi
}

\ifx\master\undefined\expandafter\autoinsert\fi

 % place at start and end of chapter
% 

\def\precommands{%
    % [ USER VARIABLES ]

\def\PHDTITLE {Regime shifts in ecology and evolution}
\def\PHDAUTHOR{Carl Boettiger}
\def\PHDSCHOOL{University of California, Davis}

\def\PHDMONTH {September}
\def\PHDYEAR  {2012}
\def\PHDDEPT {Center for Population Biology}

\def\BSSCHOOL {Princeton}
\def\BSYEAR   {2007}

\def\PHDCOMMITTEEA{Alan Hastings}
\def\PHDCOMMITTEEB{Peter Wainwright}
\def\PHDCOMMITTEEC{Brian Moore}

% [ GLOBAL SETUP ]

\documentclass[letterpaper,oneside,11pt]{report}


% [ CARL BOETTIGER`S CUSTOM COMMANDS, LIBRARIES, ETC ]

\usepackage{subfigure}
\usepackage[sort&compress]{natbib}
\usepackage{color}
\usepackage{fancyvrb}
\usepackage{ctable}


\usepackage{silence}
\WarningFilter{amsmath}{Underfull}     

%\newcommand{\argmax}{\operatorname{argmax}}
\newcommand{\ud}{\mathrm{d}}


\usepackage{calc}
\usepackage{breakcites}
\usepackage[newcommands]{ragged2e}
\usepackage{appendix}
\usepackage{comment}
\usepackage{xifthen}

\usepackage{graphicx}
\usepackage{epstopdf}


\renewenvironment{abstract}{\chapter*{Abstract}}{}
\renewcommand{\bibname}{Bibliography}
\renewcommand{\contentsname}{Table of Contents}

\makeatletter
\renewcommand{\@biblabel}[1]{\textsc{#1}}
\makeatother

% [ FONT SETTINGS ]

\usepackage[T1]{fontenc}
\usepackage{libertine}

\usepackage[tbtags, intlimits, namelimits]{amsmath}
\usepackage{amsthm}
\usepackage{amssymb}
\usepackage{amsfonts}



% [ PAGE LAYOUT ]

\usepackage{geometry}
\geometry{lmargin = 1.5in}
\geometry{rmargin = 1.0in}
\geometry{tmargin = 1.0in}
\geometry{bmargin = 1.0in}

% [ PDF SETTINGS ]

\usepackage[final]{hyperref}
\hypersetup{
    breaklinks  = {true},
    colorlinks  = {true},
    linktocpage = {false},
    linkcolor   = {blue},
    citecolor   = {black},
    urlcolor    = {black},
    plainpages  = {false},
    pageanchor  = {true},
    pdfauthor   = {\PHDAUTHOR},
    pdftitle    = {\PHDTITLE},
    pdfsubject  = {Dissertation, \PHDSCHOOL},
    pdfcreator  = {},
    pdfkeywords = {},
    pdfproducer = {}
}
\urlstyle{same}

% [ LETTER SPACING ]

\usepackage[final]{microtype}
\microtypesetup{protrusion=compatibility}
\microtypesetup{expansion=false}

\newcommand{\upper}[1]{\MakeUppercase{#1}}
\let\lsscshape\scshape

\ifcase\pdfoutput\else\microtypesetup{letterspace=15}
\renewcommand{\scshape}{\lsscshape\lsstyle}
\renewcommand{\upper}[1]{\textls[50]{\MakeUppercase{#1}}}\fi

% [ LINE SPACING ]

\usepackage[doublespacing]{setspace}
\renewcommand{\displayskipstretch}{0.75}

\setlength{\parskip   }{0em}
\setlength{\parindent }{2em}

% [ TABLE FORMATTING ]

\usepackage{booktabs}
\usepackage{multirow}
\usepackage{dcolumn}
\setlength{\heavyrulewidth}{1.5\arrayrulewidth}
\setlength{\lightrulewidth}{1.0\arrayrulewidth}
\setlength{\doublerulesep }{2.0\arrayrulewidth}

% [ SECTION FORMATTING ]

\usepackage[largestsep,nobottomtitles*]{titlesec}
\renewcommand{\bottomtitlespace}{0.75in}

\titleformat{\chapter}[display]%
    {\bfseries\huge\singlespacing}%
    {\filleft\textsc{\LARGE \chaptertitlename\ \thechapter}}%
    {-0.2ex}{\titlerule[3pt]\vspace{0.2ex}}[]

\titleformat{\section}{\LARGE}{\thesection\hspace{0.5em}}{0ex}{}
\titleformat{\subsection}{\Large}{\thesubsection\hspace{0.5em}}{0ex}{}
\titleformat{\subsubsection}{\large}{\thesubsubsection\hspace{0.5em}}{0ex}{}

\titlespacing*{\chapter}{0em}{6ex}{4ex plus 2ex minus 0ex}
\titlespacing*{\section}{0em}{2ex plus 3ex minus 1ex}{0.5ex plus 0.5ex minus 0.5ex}
\titlespacing*{\subsection}{0ex}{2ex plus 3ex minus 1ex}{0ex}
\titlespacing*{\subsubsection}{0ex}{2ex plus 0ex minus 1ex}{0ex}

% [ HEADER SETTINGS ]

\usepackage{fancyhdr}

\setlength{\headheight}{\normalbaselineskip}
\setlength{\footskip  }{0.5in}
\setlength{\headsep   }{0.5in-\headheight}

\fancyheadoffset[R]{0.5in}
\renewcommand{\headrulewidth}{0pt}
\renewcommand{\footrulewidth}{0pt}

\newcommand{\pagebox}{\parbox[r][\headheight][t]{0.5in}{\hspace\fill\thepage}}

\newcommand{\prelimheaders}{\ifx\prelim\undefined\renewcommand{\thepage}{\textit{\roman{page}}}\fancypagestyle{plain}{\fancyhf{}\fancyfoot[L]{\makebox[\textwidth-0.5in]{\thepage}}}\pagestyle{plain}\def\prelim{}\fi}

\newcommand{\normalheaders}{\renewcommand{\thepage}{\arabic{page}}\fancypagestyle{plain}{\fancyhf{}\fancyhead[R]{\pagebox}}\pagestyle{plain}}

\normalheaders{}

% [ CUSTOM COMMANDS ]

\newcommand{\signaturebox}[1]{\multicolumn{1}{p{4in}}{\vspace{3ex}}\\\midrule #1\\}

% Redefine AMS proof environment to have itshape
% Note: This environment automatically adds \qed at the end. If your proof
% ends in a math environment, the \qed is placed, undesirably, on a new line.
% To prevent that, insert \qedhere inside the math environment.
\makeatletter
\renewenvironment{proof}[1][\proofname]{%
\par\pushQED{\qed}\normalfont%
\topsep6\p@\@plus6\p@\relax\trivlist%
\item[\hskip\labelsep\bfseries#1\@addpunct{.}]\itshape\ignorespaces}{%
\popQED\endtrivlist\@endpefalse}%
\makeatother

% TUGboat, Volume 0 (2001), No. 0
% http://math.arizona.edu/~aprl/publications/mathclap/perlis_mathclap_24Jun2003.pdf
% For comparison, here are the existing overlap macros:
% \def\llap#1{\hbox to 0pt{\hss#1}}
% \def\rlap#1{\hbox to 0pt{#1\hss}}
\def\clap#1{\hbox to 0pt{\hss#1\hss}}
\def\mathllap{\mathpalette\mathllapinternal}
\def\mathrlap{\mathpalette\mathrlapinternal}
\def\mathclap{\mathpalette\mathclapinternal}
\def\mathllapinternal#1#2{%
\llap{$\mathsurround=0pt#1{#2}$}}
\def\mathrlapinternal#1#2{%
\rlap{$\mathsurround=0pt#1{#2}$}}
\def\mathclapinternal#1#2{%
\clap{$\mathsurround=0pt#1{#2}$}}

\newcommand{\alert}[1]{\textbf{\textcolor{red}{#1}}}




% [ Code blocks ]
%\DefineShortVerb[commandchars=\\\{\}]{\|}
\DefineVerbatimEnvironment{Highlighting}{Verbatim}{commandchars=\\\{\}}
% Add ',fontsize=\small' for more characters per line
\newenvironment{Shaded}{}{}
\newcommand{\KeywordTok}[1]{\textcolor[rgb]{0.00,0.44,0.13}{\textbf{{#1}}}}
\newcommand{\DataTypeTok}[1]{\textcolor[rgb]{0.56,0.13,0.00}{{#1}}}
\newcommand{\DecValTok}[1]{\textcolor[rgb]{0.25,0.63,0.44}{{#1}}}
\newcommand{\BaseNTok}[1]{\textcolor[rgb]{0.25,0.63,0.44}{{#1}}}
\newcommand{\FloatTok}[1]{\textcolor[rgb]{0.25,0.63,0.44}{{#1}}}
\newcommand{\CharTok}[1]{\textcolor[rgb]{0.25,0.44,0.63}{{#1}}}
\newcommand{\StringTok}[1]{\textcolor[rgb]{0.25,0.44,0.63}{{#1}}}
\newcommand{\CommentTok}[1]{\textcolor[rgb]{0.38,0.63,0.69}{\textit{{#1}}}}
\newcommand{\OtherTok}[1]{\textcolor[rgb]{0.00,0.44,0.13}{{#1}}}
\newcommand{\AlertTok}[1]{\textcolor[rgb]{1.00,0.00,0.00}{\textbf{{#1}}}}
\newcommand{\FunctionTok}[1]{\textcolor[rgb]{0.02,0.16,0.49}{{#1}}}
\newcommand{\RegionMarkerTok}[1]{{#1}}
\newcommand{\ErrorTok}[1]{\textcolor[rgb]{1.00,0.00,0.00}{\textbf{{#1}}}}
\newcommand{\NormalTok}[1]{{#1}}



    \def\preinserted{}
    \begin{document}
}

\def\postcommands{%
    \singlespacing
    \phantomsection
%    \bibliographystyle{../bibliography/expanded}
   \bibliographystyle{elsarticle-harv}
   \bibliography{../bibliography/references}
    \enddocument
}

\def\autoinsert{%
    \ifx\preinserted\undefined
        \expandafter\precommands
    \else
        \expandafter\postcommands
    \fi
}

\ifx\master\undefined\expandafter\autoinsert\fi



